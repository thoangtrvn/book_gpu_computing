%%
%% Fortran_IO.tex
%% Login : <hoang-trong@hoang-trong-laptop>
%% Started on  Mon Jul 27 15:03:37 2009 Hoang-Trong Minh Tuan
%% $Id$
%% 
%% Copyright (C) 2009 Hoang-Trong Minh Tuan
%%

\chapter{Fortran I/O}
\label{chap:fortran-io}


\section{Mechanism}
\label{sec:mechanism}

There are different ways to accomplish I/O tasks in Fortran. Data is
transferred in and out between the program and devices or files
through a Fortran {\bf logical unit}. A logical unit is an integer
number of 4-byte, i.e. in the range 0..2,147,483,647.

The character * can also be identified as a logical unit, and is used
for standard input/output depending on where it appears, e.g. READ or
WRITE.

To work with a specific, named file, you need to use the OPEN
statement to assign a logical unit to that named file. 

\section{Default I/O}
\label{sec:default-io}


\section{Accessing a named file}
\label{sec:accessing-named-file}

\url{http://docs.sun.com/source/819-3685/2_io.html}

\url{http://www.ibiblio.org/pub/languages/fortran/ch2-14.html}

%%% Local Variables: 
%%% mode: latex
%%% TeX-master: "gpucomputing"
%%% End: 
