
\chapter{cuFFT}
\label{chap:cufft}

% \section{cuFFT}
% \label{sec:cufft}

\section{Introduction}
\label{sec:introduction-2}


Fast Fourier Transform is a devide-and-conquer algorithm for
efficiently computing discrete Fourier transform of complex or
real-valued data sets. cuFFT is the CUDA implementation of the FFT
library. 

It can do
\begin{enumerate}
\item 1D, 2D and 3D transforms of complex and real-valued data
\item Batched execution for doing multiple 1D transforms in parallel
\item 1D transform size up to 8M elements
\item 2D and 3D transform sizes in the range [2,16384]
\item In-place and out-of-place transforms for real and complex data.
\end{enumerate}

Some functions:
\begin{enumerate}
\item real and complex transform: \verb!cufft_C2C!, \verb!cufft_C2R!,
  \verb!cufft_R2C!
\item direction of transform: \verb!cufft_FORWARD (-1)! and
  \verb!cufft_INVERSE (1)!
\item a complex type is defined: \verb!cufftComplex!
\item real to complex FFT (i.e. output array holds only non-redundant
  coefficients): 
  \begin{itemize}
  \item thus N element $\rightarrow$ $N/2+1$ elements.
  \item for in-place transform, the input/output arrays need to be
    padded. 
  \end{itemize}
\item un-normalized transform: \verb!IFFT(FFT(A))= length(A)*A!
\end{enumerate}

{\bf NOTE}: For 2D and 3D transform, cuFFT use row-major (C-order).
Thus, if calling from FORTRAN or MATLAB, remember to change the order
of size parameters during plan creation.

CUFFT APIis is modeled after FFTW.

{\bf NOTE}: Once a plan is created, the library stores whatever state
is needed to execute the plan multiple times without recomputing the
configuration.


\begin{lstlisting}
#define NX 256
#define NY 128

cufftHandle plan;
cufftComplex *idata, *odata;
cudaMalloc((void**)&idata, sizeof(cufftComplex)*NX*NY);
cudaMalloc((void**)&odata, sizeof(cufftComplex)*NX*NY);
 :
/* Create a 1D FFT plan. */
cufftPlan2d(&plan, NX,NY, CUFFT_C2C);
/* Use the CUFFT plan to transform the signal out of place. */
cufftExecC2C(plan, idata, odata, CUFFT_FORWARD);
/* Inverse transform the signal in place. */
cufftExecC2C(plan, odata, odata, CUFFT_INVERSE);
/* Note:
Different pointers to input and output arrays implies 
out of place transformation
*/
/* Destroy the CUFFT plan. */
cufftDestroy(plan);
cudaFree(idata), cudaFree(odata);
\end{lstlisting}


%%% Local Variables: 
%%% mode: latex
%%% TeX-master: "gpucomputing"
%%% End: 
