
\chapter{Partial Differential Equations}
\label{chap:part-diff-equat}

\section{Introduction}
\label{sec:introduction-9}

{\bf Partial differential equations} (PDE) is a powerful method that
has been intensively utilized in computer simulation of continuous
systems. PDE are classified into 3 categories:
\begin{itemize}
\item {\it hyperbolic}: e.g. the 1D wave equation
  \begin{eqnarray}
    \label{eq:13}
    \frac{\partial^2 u}{\partial t^2} = v^2
    \frac{\partial^2u}{\partial x^2}
  \end{eqnarray}
with $v$ is the velocity
\item {\it parabolic}: e.g. the 1D diffusion equation
  \begin{eqnarray}
    \label{eq:14}
    \frac{\partial u}{\partial t} = \frac{\partial}{\partial x}\left(
    \frac{D\partial u}{\partial x}\right)
  \end{eqnarray}
with $D$ is diffusion coefficient. 
\item {\it elliptic}: e.g. Poisson equation
  \begin{eqnarray}
    \label{eq:15}
    \frac{\partial^2 u}{\partial x^2} + \frac{\partial^2 u}{\partial
      y^2}=  \rho(x,y)
  \end{eqnarray}
with $\rho$ is the source term. If $\rho=0$, it becomes {\it Laplace's
  equation}. 
\end{itemize}

\section{Hyperbolic PDE}
\label{sec:hyperbolic-pde}

A {\bf hyperbolic PDE} of order $n$ is a PDE that has a well-posed IVP
for the first $n-1$ derivatives. Many of the equations of mechanics
are hyperbolic. The model of hyperbolic equation is the
{\bf wave equation} (read Sec.~\ref{sec:wave-equation}). 
\begin{eqnarray*}
  u_{tt}-c^2u_{xx} = 0
\end{eqnarray*}

\section{Elliptic PDE}
\label{sec:elliptic-pde}


\section{Parabolic PDE}
\label{sec:parabolic-pde}

A {\bf parabolic PDE} is a second-order PDE, describing a wide family
of problems in science, e.g. heat diffusion, stock option pricing,
ocean acoustic propagation...

A PDE of the form
\begin{eqnarray*}
  Au_{xx}+Bu_{xy}+Cu_{yy}+... = 0
\end{eqnarray*}
is a parabolic if $B^2-4AC=0$.



%%% Local Variables: 
%%% mode: latex
%%% TeX-master: "gpucomputing"
%%% End: 
