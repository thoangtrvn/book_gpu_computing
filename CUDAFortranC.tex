
\chapter{Interoperability CUDA Fortran and C}
\label{chap:inter-cuda-fortr}

\begin{table}[hbt]
\begin{center}
\caption{Comparison}
\begin{tabular}{cc} 
  \hline
  CUDA C & CUDA Fortran \\ 
  \hline\hline
  support texture memory & no \\
  support runtime API & yes \\
  support driver API & no \\
  cudaMalloc, cudaFree & allocate, deallocate \\
  cudaMemcpy & using assignments \\
  OpenGL interoperability & no\\
  Direct3D interoperability & no \\
  texture memory & no \\
  array zero-based & array 1-based \\
  threadIdx/blockIdx zero-based & 1-based \\
  use unbounded pointers (for dynamic data) & allocatable are device/host \\
  pinned allocate routines & pinned attribute \\
\hline
\end{tabular}
\end{center}
\label{tab:CUDA_F_C}
\end{table}

From Fortran, we can call CUDA C kernel by defining explicit interface
with \verb!bind(C)!
\begin{lstlisting}
use cudafor

interface
  attributes(global) subroutine saxpy(a, x, y, n) bind(c)
    real, device :: x(*), y(*)
    real, value :: a
    integer, value :: n
  end subroutine
end interface

....
call saxpy<<<grid, blocks>>>(aa, xx, yy, nn)
\end{lstlisting}

From C we can call CUDA Fortran kernels
\begin{lstlisting}
extern __global__ void saxpy_ ( float a, float*x, float*y, int n) ;;

...
saxpy_ (a, x, y, n);
\end{lstlisting}
with the function in Fortran is
\begin{lstlisting}
attributes (global) subroutine saxpy(a, x, y, n)
  real, value :: a
  real :: x(*), y(*)
  integer, value :: n
 ....

end subroutine
\end{lstlisting}
There are important to notice
\begin{enumerate}
\item the name is right, i.e. one underscore behind
  (\verb!subroutine_!, or \verb!module_subroutine_!)
\item pass value vs. reference arguments
\end{enumerate}


If you're working directly with source code, 
\begin{enumerate}
 \item Fortran CUDA kernel can be linked with \verb!nvcc!
\item C CUDA kernel can also be linked with \verb!pgfortran!, using
  \verb!-Mcuda! flag. 
\end{enumerate}




%%% Local Variables: 
%%% mode: latex
%%% TeX-master: "gpucomputing"
%%% End: 
