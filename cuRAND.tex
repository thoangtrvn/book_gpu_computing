\chapter{CURAND (CUDA Random Number Generator)}
\label{chap:cuRAND}

\section{introduction}

cuRAND is an nVIDIA library designed to generate pseudo-random numbers (1D
dimension), and quasi-random numbers (multidimensional space). The algorithms
are deterministic. So, for a given seed value, we always get the same sequences.
Depending on how long of your simulation, you may expect the sequence period to
be long enough so that it's not repeated in your simulation. 

cuRAND APIs can be used to generate random numbers on CPU side or GPU side,
depending on the APIs you use. 
\begin{verbatim}
#include </include/curand.h>
\end{verbatim}
For random numebers generated on GPU side, they're stored on global device
memory. 
\begin{verbatim}
#include </include/curand_kernel.h>
\end{verbatim}

Using different versions of curand.h and shared library is not supported. So,
it's important to know which version of CUDA you're using. 
\begin{enumerate}
  \item quasi-random Sobol supports 32-bit and 64-bit with upto 20,000
  dimensions.
\end{enumerate}

What to do?
\begin{enumerate}
  \item \verb!curandCreateGenerator()! create a generator (GPU side) or
  \verb!curandCreateGeneratorHost()! (CPU side)
  \item \verb!curandSetPseudoRandomGeneratorSeed()! set a seed value
  \item Choose the distribution to generate data
  \begin{verbatim}
  curandGenerateUniform()/(curandGenerateUniformDouble(): Uniform
curandGenerateNormal()/cuRandGenerateNormalDouble(): Gaussian
curandGenerateLogNormal/curandGenerateLogNormalDouble(): Log-Normal
  \end{verbatim}
  
  NOTE: the output pointer you pass to curandGenerateUniform() will be a pointer
  to host memory allocated with malloc() for the generator created with
  curandCreateGeneratorHost(). For a generator created with
  curandCreateGenerator() the pointer will be to device memory allocated with cudaMalloc().  
  
  \item \ldots 
  \item When you're finished, call \verb!curandDestroyGenerator()! to destroy
  the generator. 
\end{enumerate}

IMPORTANT: Using the same seed and the same method, but the generated sequences
on CPU and GPU are the same if using the generator. Using device APIs are not
guaranteed. The only way to guarantees you always get the same sequence is to
generate from CPU side and copy to GPU \footnote{\url{http://forums.nvidia.com/index.php?showtopic=191089}}. 


If you want to generate a random number directly from inside the device kernel
(i.e. inline generation), you can use device API like \verb!curand_init()! and
\verb!curand_uniform()!. In this scenario, the generated random numbers can be directly stored in the
register, rather than global device memory. There is an "unsupported" way
(i.e. not documented and not fully tested) to get CPU versions of
\verb!curand_init()! and \verb!curand_uniform()!. To get CPU version of the
device APIs, we can do
\begin{verbatim}
#define QUALIFIERS __host__ __device__
#include <curand_kernel.h>
#undef QUALIFIERS
\end{verbatim}

Using \verb!curand_init()! and \verb!curand()! from the device API gives you one
stream of random numbers. Using \verb!curandGenerate()! launches 4096 threads on
the device, each one of which is doing \verb!curand_init()! and then lots of
calls to curand().



\section{cuRAND on CUDA 3.2}
\label{sec:cuRAND_CUDA3.2}

It supports Sobol quasi-random and XORWOW pseudo-random number generators.
  
\subsection{Sobol}

\subsection{XORWOW}

The precomputed arrays are declared as \verb!__constant__! tables in
\verb!curand_precalc.h!.


\section{cuRAND on CUDA 4.1}
\label{sec:cuRAND_CUDA4.1}

\subsection{MRG32k3a}
\label{sec:MRG32k3a}

L'Ecuyer's Multiple Recursive Generator
is the one that used Combined Multiple Recursive Generators (CMRG) in order to
produce a generator that had good randomness properties with a long
period, while at the same time being fairly straightforward to implement.

 

The library need to use some pre-computed matrices, declared in the file
\verb!curand_mrg32k3a.h!. So we need to include this file all the time we use
MRG32k3a. 

Prior calling the init, we need to copy the arrays (matrix) to device memory and
fill in a structure of type \verb!curandMRG32k3aPtrs_t! with the device pointers to the
corresponding arrays, and copy that structure to device memory.

\begin{verbatim}
 curandMRG32k3aPtrs_t MRG;
        curandMRG32k3aPtrs_t gpu_scratch_area;
        int * MRGTables

        if(cudaMalloc((void **)&(MRG.unitsM1),
                        (2 * sizeof(struct sMRG32k3aSkipUnits)) + 
                        (2 * sizeof(struct sMRG32k3aSkipSubSeq)) + 
                        (2 * sizeof(struct sMRG32k3aSkipSeq)) +
                        sizeof(curandMRG32k3aPtrs_t) )!= cudaSuccess)
        {
            return CURAND_STATUS_ALLOCATION_FAILED;
        }
        if(cudaMemcpy(MRG.unitsM1,
                      ((char *)(&mrg32k3aM1[0][0][0])),
                      sizeof( struct sMRG32k3aSkipUnits ), 
                      cudaMemcpyHostToDevice) != cudaSuccess)
        {
            cudaFree(MRG.unitsM1);
            return CURAND_STATUS_INITIALIZATION_FAILED;
        }
        MRG.unitsM2 = (sMRG32k3aSkipUnits_t *)((char *)MRG.unitsM1 + 
                        sizeof( struct sMRG32k3aSkipUnits ));
        if(cudaMemcpy(MRG.unitsM2, 
                      &mrg32k3aM2[0][0][0],
                      sizeof( struct sMRG32k3aSkipUnits ), 
                      cudaMemcpyHostToDevice) != cudaSuccess)
        {
            cudaFree(MRG.unitsM1);
            return CURAND_STATUS_INITIALIZATION_FAILED;
        }
        MRG.subSeqM1 = (sMRG32k3aSkipSubSeq_t *)((char *)MRG.unitsM2 + 
                        sizeof( struct sMRG32k3aSkipUnits ));
        if(cudaMemcpy(MRG.subSeqM1, 
                      &mrg32k3aM1SubSeq[0][0][0],
                      sizeof( struct sMRG32k3aSkipSubSeq ), 
                      cudaMemcpyHostToDevice) != cudaSuccess)
        {
            cudaFree(MRG.unitsM1);
            return CURAND_STATUS_INITIALIZATION_FAILED;
        }
        MRG.subSeqM2 = (sMRG32k3aSkipSubSeq_t *)((char *)MRG.subSeqM1 + 
                        sizeof( struct sMRG32k3aSkipSubSeq ));
        if(cudaMemcpy(MRG.subSeqM2, 
                      &mrg32k3aM2SubSeq[0][0][0],
                      sizeof( struct sMRG32k3aSkipSubSeq ), 
                      cudaMemcpyHostToDevice) != cudaSuccess)
        {
            cudaFree(MRG.unitsM1);
            return CURAND_STATUS_INITIALIZATION_FAILED;
        }
        MRG.seqM1 = (sMRG32k3aSkipSeq_t *)((char *)MRG.subSeqM2 + 
                     sizeof( struct sMRG32k3aSkipSubSeq ));
        if(cudaMemcpy(MRG.seqM1, 
                      &mrg32k3aM1Seq[0][0][0],
                      sizeof( struct sMRG32k3aSkipSeq ), 
                      cudaMemcpyHostToDevice) != cudaSuccess)
        {
            cudaFree(MRG.unitsM1);
            return CURAND_STATUS_INITIALIZATION_FAILED;
        }
        MRG.seqM2 = (sMRG32k3aSkipSeq_t *)((char *)MRG.seqM1 + 
                     sizeof( struct sMRG32k3aSkipSeq ));
        if(cudaMemcpy(MRG.seqM2, 
                      &mrg32k3aM2Seq[0][0][0],
                      sizeof( struct sMRG32k3aSkipSeq ), 
                      cudaMemcpyHostToDevice) != cudaSuccess)
        {
            cudaFree(MRG.unitsM1);
            return CURAND_STATUS_INITIALIZATION_FAILED;
        }
        gpu_scratch_area = (unsigned int *)((char *)MRG.seqM2 + 
                               sizeof( struct sMRG32k3aSkipSeq ));
        if(cudaMemcpy(gpu_scratch_area, 
                      &MRG,
                      sizeof( curandMRG32k3aPtrs_t ), 
                      cudaMemcpyHostToDevice) != cudaSuccess)
        {
            cudaFree(MRG.unitsM1);
            return CURAND_STATUS_INITIALIZATION_FAILED;
        }
        MRGTables = (int *)MRG.unitsM1;
\end{verbatim}

References:
\begin{itemize}
  \item \url{http://forums.nvidia.com/index.php?showtopic=224938}
\end{itemize}
