%%
%% CodingConvention.tex
%% Login : <hoang-trong@hoang-trong-laptop>
%% Started on  Thu Jul 23 13:44:41 2009 Hoang-Trong Minh Tuan
%% $Id$
%% 
%% Copyright (C) 2009 Hoang-Trong Minh Tuan
%%

\chapter{Coding conventions}
\label{chap:coding-conventions}

\section{Filename}
\label{sec:filename}

A module should be given a good name
\begin{itemize}
\item new data structure, e.g. matrix, vector... should be in the file
  with the same name, e.g. matrix.f95, vector.f95

\item utility function should be classified into its appropriate group
  with the filename ending in \lstinline!_util.f95!,
  e.g. \lstinline!math_util.f95!
\end{itemize}

\section{Variables}
\label{sec:variables}
NOTE:

Length should be less than 31.

Array variables start with \lstinline!a_!

Vector variables start with \lstinline!v_!

Matrix variables start with \lstinline!m_!

Integer type ends with \verb!_i!

Real type ends with  \verb!_r!

Variables with multiple words should be separated by underscore (\_).

\section{Methods}
\label{sec:methods}

Methods are lower case connected with underscores, i.e. init\_fruit,
failed\_assert\_count. Methods should begin with a verb, e.g. get,
set, calculate...

All methods (written in a module) must be exposed by interface
\begin{lstlisting}
MODULE 
  INTERFACE init_data
CONTAINS
 ...
END MODULE
\end{lstlisting}

\section{Lines}
\label{sec:lines}

Keep each line less than 132 characters, however 80 is the recommended
limit.

\section{Programming Style}
\label{sec:programming-style}

\begin{enumerate}
\item Use \verb!IMPLICIT NONE!
\item Use \verb!DO ENDDO! rather than DO-loops
\item Use variable names with more than 6 characters and/or underscore
\item Initialize variables before using it
\end{enumerate}


%%% Local Variables: 
%%% mode: latex
%%% TeX-master: "gpucomputing"
%%% End: 
