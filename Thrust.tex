
\chapter{Thrust}
\label{chap:thrust}


\section{Introduction}
\label{sec:introduction-3}


Thrust is a CUDA-capable library written in C, yet with C++ template
interface. % Thus, it's better to use with CUDA from version 3.0.
Latest Thrust 1.3 support CUDA 3.2. 
You need to download the file, unzip and copy the full folder named
\verb!thrust! to the folder /usr/local/cuda/include.

Thrust library provides two vector containers 
\begin{itemize}
\item \verb!host_vector!: store objects on the host memory
\item \verb!device_vector!: store objects on the device memory (by reference)
\end{itemize}
Just like \verb!std:vector! in C++ STL, both behave like generic
containers, i.e. store any data type and can be resized dynamically.
Thrust can do several things, and one of them is sorting.

For primitive data types (char, int, float, double, long), with the
default comparison method \verb!thrust::less<T>! is used (sort
ascending), then Thrust use radix-sort.

\begin{lstlisting}
#include <thrust/host_vector.h>
#include <thrust/device_vector.h>

int main (void) {
  thrust::host_vector<int> H(5); // vector of 5 elements

  H.resize(2) ;; 

  thrust::device_vector<int> D = H; //copy host to device vector
                      // D has 2 elements
  // = operator can be used to copy a host data to device data
  // elements can be accesssed uing [] notation (this is, however, not 
  // + recommended as it call to cudaMemcpy()
  D[0] = 99;
  D[1] = 23;

  thrust::host_vector<int> D(10,2) ; //initialize all 10 elements to 2

  //set the first 7  elements to 9
  thrust::fill(D.begin(), D.begin() + 7, 9) ; 

  //set the elements of H to 0,1,2,3
  thrust::sequence(H.begin(), H.end());

  //copy data in H to the beginning of D
  thrust::copy(H.begin(), H.end(), D.begin());

  return 0; // no need to free 
}
\end{lstlisting}

All methods use {\bf static dispatching}, thus it determines the
appropriate operations at compile-time. It implies that there is no
runtime overhead to the dispatch process.

\subsection{Thrust 1.3}
\label{sec:thrust-1.3}

Thrust 1.3 has new/improved features
\begin{itemize}
\item radix sorting (fixed-length keys): keys can be of any C/C++
  intrinsic types (signed char, float, unsigned long long...) with 1B
  integer keys/sec in C2050, and 750 M integer keys/sec. With
  floating-point keys, it has 0.5-1.5\%
  overhead\footnote{\url{http://code.google.com/p/back40computing/wiki/RadixSorting}}
  . 
\item 
\end{itemize}

NOTE: Thrust 1.3 supports up to CUDA 3.2 and remove the support for
CUDA 2.3. 

\subsection{Thrust 1.2.1}
\label{sec:thrust-1.2.1}

Thrust 1.2.1 support CUDA 3.1. 

\section{Raw pointers with Thrust}

You can pass raw pointers to Thrust functions and it dispatch the host
path of the algorithm by default. If you want to specify the pointer
as pionters to device memory, you need to wrap it with
\verb!thrust::device_ptr! before calling to the function.

Thus, you don't use \verb!host_vector! nor \verb!device_vector! to
define a variable, instead you use \verb!device_ptr!
\begin{lstlisting}
size_t N = 10;

int *raw_ptr; 
cudaMalloc((void **) &raw_ptr, N *sizeof(int));

thrust::device_ptr<int> dev_ptr(raw_ptr);

thrust::fill(dev_ptr, dew_ptr + N, (int) 0);
\end{lstlisting}

To extract a raw pointer from a \verb!device_ptr! you use
\verb!raw_pointer_cast! function
\begin{lstlisting}
size_t N = 10;

thrust::device_ptr<int> dev_ptr = thrust::device_malloc<int> N;

//extract raw pointer
int *raw_ptr  = thrust::raw_pointer_cast(dev_ptr);

\end{lstlisting}

\section{Sort}

\subsection{CPU}

Thrust provides two analogs of STL's \verb!std::sort! and
\verb!std::stable_sort!, with ascending by default (i.e. using $<$ operator).

\begin{enumerate}
\item  \verb!thrust::sort!
\item \verb!thrut::stable_sort!
\item \verb!thrust:sort_by_key!
\item \verb!thrust:stable_short_by_key!
\end{enumerate}

\begin{lstlisting}
#include <thrust/sort.h>
const int N=5;
int keys[N] = {1,4,2,7,5};
char values[N] = {'a', 'b', 'c', 'd', 'e'};

thrust::short_by_key(keys, keys+N, values);

int A[N] = {1,4,3,8,5};
thrust::stable_sort(A, A+N, thrust::greater<int>());

thrust::sort(A, A+N, thrust::greater<int>());
\end{lstlisting}

You are allowed to provide a comparison function as the third argument
to \verb!thrust::sort()! or \verb!thrust::stable_sort()! just like
STL's std::sort function. 

\subsection{GPU}

On CUDA 4.0, parallel sorting improve 5x to 100x faster.

\section{Thrust on CUDA 4.0}
\label{sec:thrust_cuda40}

With CUDA 4.0, Thrust introduced new data structure (similar to that of C++
STL)
\begin{lstlisting}
thrust:device_vector
thrust:host_vector
thrust:device_ptr  
...
\end{lstlisting}
and new algorithms
\begin{lstlisting}
thrust:sort
thrust:reduce
thrust:exclusive_scan  
\end{lstlisting}

Thrust can also choose either using GPU or multi-core CPU whichever is tha
fastest code-path at compile time. 



References:
\begin{itemize}
\item \url{http://code.google.com/p/thrust/wiki/Tutorial}
\item \url{http://code/google/com/p/thrust/}
\end{itemize}



%%% Local Variables: 
%%% mode: latex
%%% TeX-master: "gpucomputing"
%%% End: 
