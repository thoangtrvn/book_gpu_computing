\chapter{CUDA C++}
\label{chap:CUDA_C++}

\section{C++ support}
\label{sec:c++-support}

Remember that CUDA C is an extention to the C language, with some additional
access specifiers (Sect.\ref{sec:kernels}). CUDA host code has been compiled as
C++ code (i.e. \verb!-ccbin=g++!) since version 2.

Since Fermi architecture, CUDA supports full C++, with class and other
object-oriented features (e.g. template). C++ 11 features supported in host and
device code since CUDA 7.
\begin{itemize}
  \item  \verb!auto! keyword inside CUDA kernel
  \item \verb!template! for CUDA kernel - Sect.\ref{sec:template-CUDA-kernel}
  
  \item memory management
  
  \item using lambdas inside CUDA kernel
  
  \item range-based for loops inside CUDA kernel
\end{itemize}

It means that we can allocate an array of objects on GPU memory who are instance
of a C++ class.
\begin{enumerate}
  
  \item  In order to evoke the method of an object from a CUDA \verb!__global__!
  kernel (and of course from also host-side), the method
needs to be defined with both
\begin{verbatim}
__host__  __device__ 
\end{verbatim}
declspecs (declaration specifiers). Also, we need to use that for 
the constructor and destructor if you plan to use new/delete on the device

NOTE: Use the above for the constructor and destructor as well, if we want to
use  \verb!new!/\verb!delete! operator on that C++ class to dynamically create
an object within CUDA kernel (\verb!__global__! or \verb!__device__!). This feature requires
CUDA 4.0 and a compute capability 2.0 or higher  GPU).

  \item Programmers can use C++ virtual functions, function pointers, dynamic object
allocation, try/catch statements (which handle exceptions), and make system
calls, such as stdio.h (the standard input/output channel). NVIDIA provides a
CUDA C/C++ compiler, but any third-party tool vendor can target the CUDA
architecture.

  
  \item Compiling the header file + source file using \verb!nvcc! compiler.
  
  The reason for this is that only the CUDA compiler knows \verb!__device__! and
  \verb!__host__! -- your host C++ compiler will raise an error.
\end{enumerate}

\begin{mdframed}

Fermi expands the memory-addressing range from 32 bits (maximum
4GB of memory) to 40 bits (maximum 1TB of memory).

``{\it A key factor in bringing C++ to CUDA is Fermi's new unified virtual
  memory addressing (UVA - Sect.\ref{sec:UVA}). 
  
  Previous CUDA architectures use direct memory addresses and thus allocated
  different memory regions for local memory, shared memory, and global memory.
  The fragmented memory map required different load/store instructions for each
  region and prevented NVIDIA from fully implementing pointers in C and C++. 
  
  In particular, C++ relies heavily on memory pointers, because every object in
  this object-oriented language requires at least one pointer.  A typical
  program creates hundreds of objects.
  The Fermi architecture fixes those problems by unifying everything in a single
  memory space} ''


Besides bringing C++ to CUDA, the Fermi architecture with PTX 2.0 makes it
easier to use other high level programming languages and compilers. Fermi
supports FORTRAN  ``a vital language for scientific and engineering applications
'' as well as Java (via native methods) and Python. Many financial programs are
written in Java, and Python is popular for rapid application development. PTX
2.0 also adds features for OpenCL (Sect.\ref{sec:OpenCL}) and DirectCompute
(Sect.\ref{chap:DirectCompute}), such as new bit-reverse and append
instructions.
\end{mdframed}

\verb!__CUDACC__! macro - Sect.\ref{sec:__CUDACC__-macro}
\begin{verbatim}
#ifdef __CUDACC__
#define CUDA_CALLABLE_MEMBER __host__ __device__
#else
#define CUDA_CALLABLE_MEMBER
#endif 


class Foo {
public:
    CUDA_CALLABLE_MEMBER Foo() {}
    CUDA_CALLABLE_MEMBER ~Foo() {}
    CUDA_CALLABLE_MEMBER void aMethod() {}
};
\end{verbatim}




\subsection{\_\_CUDA\_ARCH\_\_ macro}
\label{sec:__CUDA_ARCH__}

The architecture identification macro \verb!__CUDA_ARCH__! is assigned a
three-digit value string \verb!xy0! (ending in a literal 0) during each nvcc
compilation stage 1 that compiles for \verb!arch=compute_xy!.

WHEN TO USE?  can be used in the implementation of GPU functions for determining
the virtual architecture for which it is currently being compiled.


WHEN NOT TO USE? 
\begin{enumerate}
  \item do not use the macro inside the body of \verb!__global__! functions/function templates.
  
  \item if it is used to control the instantiate of a \verb!__global__! function template
  
  \item do not use to select the type of \verb!__device__! and \verb!__constant__! variables
  
\begin{lstlisting}
#if !defined(__CUDA_ARCH__)
typedef int mytype;
#else
typedef double mytype;
#endif

__device__ mytype xxx;         // error: xxx's type depends on __CUDA_ARCH__
__global__ void foo(mytype in, // error: foo's type depends on __CUDA_ARCH__
                    mytype *ptr)
{
  *ptr = in;
}
\end{lstlisting}  

  \item do not use with textures and surfaces data

    
  \item  must not be used in headers if different objects could contain different behavior.
  
  Example: If the macro is used in the header file, 
\begin{lstlisting}
// header-file
template<typename T>
__device__ T* getptr(void)
{
#if __CUDA_ARCH__ == 500
  return NULL; /* no address */
#else
  __shared__ T arr[256];
  return arr;
#endif
}
\end{lstlisting} 
  and the header file is
  included by different source files, and these source files are compiled into a different compute arch.
\begin{verbatim}


nvcc --gpu-architecture=compute_50 --device-c a.cu
nvcc --gpu-architecture=compute_52 --device-c b.cu
nvcc --gpu-architecture=sm_52 a.o b.o
\end{verbatim}
  If a weak function or template function is defined in a header and its
  behavior depends on \verb!__CUDA_ARCH__!, then the instances of that function in the
  objects could conflict if the objects are compiled for different compute arch.
  
  IMPORTANT: If use, it must be guaranteed that all objects will compile for the
  same \verb!compute_arch!.
  
  
  
  \item The host code (the non-GPU code) must not depend on it.
\end{enumerate}

\subsection{\_\_global\_\_ method?}

CUDA does NOT allow defining a \verb!__global__! method of a class, regardless if the method is static or not.

So, to define a \verb!__global__! method, it has to be a global scope method, outside any class definition.

\subsection{\_\_device\_\_ method}

CUDA allows defining a method as \verb!__device__!, optionally with
\verb!__host__! (so that the method can be used on host-side or gpu-side).

IMPORTANT: If a \verb!__device__! method is a \verb!virtual! function, then its
virtual table is only available on GPU; and thus it cannot be used on host-side,
even if we use \verb!__host__! specifier. Because of this limitation, a virtual
\verb!__device__! \verb!__host__! function can only be used on device-side.


\subsection{Features on C++ extension that cannot be used on device code}


We cannot use on device code the following
\begin{verbatim}
__int128 and _Complex types
\end{verbatim}

LIMITATION: in host code on 64-bit x86 Linux platforms.
\begin{verbatim}
__float128
\end{verbatim}

\subsection{Where to use \_\_device\_\_, \_\_shared\_\_, \_\_managed\_\_ and \_\_constant\_\_ }

\begin{verbatim}
__device__, __shared__, __managed__ and __constant__ 
\end{verbatim}
cannot be used
\begin{enumerate}
  \item  class, struct, and union data members,
  
  \item formal parameter
  
\begin{verbatim}
 // this is a formal parameter
... some_function(__device__ int a)  //error
{
}
\end{verbatim}  

  \item non-extern variable declarations within a function that executes on the host.
  
  \item on variable declarations that are neither extern nor static within a function that executes on the device. 
  
  \item variable definition cannot have a class type with a non-empty constructor or a non-empty destructor
\end{enumerate}


\subsection{NO exception handling in device code}

Exception handling is only supported in host code, but not in device code.

Exception specification is not supported for \verb!__global__! functions.

\subsection{template CUDA kernel}
\label{sec:template-CUDA-kernel}

Instead of writing two different kernels
\begin{verbatim}
__global__ void saxpy(float alpha, float* x, float* y, size_t n){
 auto i = blockDim.x * blockIdx.x + threadIdx.x;
 if(i < n){
 y[i] = a * x[i] + y[i];
 }
}

__global__ void daxpy(double alpha, double* x, double* y, size_t n){
 auto i = blockDim.x * blockIdx.x + threadIdx.x;
 if(i < n){
 y[i] = a * x[i] + y[i];
 }
}
\end{verbatim}

Since CUDA 7.0, we can write
\begin{verbatim}
template <typename T>
__global__ void axpy(T alpha, T* x, T* y, size_t n){
 auto i = blockDim.x * blockIdx.x + threadIdx.x;
 if(i < n){
 y[i] = a * x[i] + y[i];
 }
}
\end{verbatim}

\section{Writing class in CUDA}

IMPORTANT: The method, if using both \verb!__device__! and \verb!__host__!,
must be defined in the same file with the class declaration, i.e. using a single
\verb!.h! file (or .cu file).


Any method that must be called from device code should be defined with both
\verb!__device__! and \verb!__host__! declspecs, including the constructor and
destructor if you plan to use new/delete on the device (Using \verb!new/delete!
requires CUDA 4.0+ and device with compute capability 2.0+).

A good strategy is to define a macro
\begin{verbatim}
#ifdef __CUDACC__
#define CUDA_CALLABLE_MEMBER __host__ __device__
#else
#define CUDA_CALLABLE_MEMBER
#endif 
\end{verbatim}

and use the macros on your member functions that you want to expose to the
kernel
\begin{verbatim}
class Foo {
public:
    CUDA_CALLABLE_MEMBER Foo() {}
    CUDA_CALLABLE_MEMBER ~Foo() {}
    CUDA_CALLABLE_MEMBER void aMethod() {}
};
\end{verbatim}

\url{http://stackoverflow.com/questions/6978643/cuda-and-classes}

\subsection{-- Before Unified Memory: explicit copy individual data elements}

\begin{verbatim}
struct dataElem {
 int prop1;
 int prop2;
 char *text;
};
\end{verbatim}
We need two copies
\begin{verbatim}
CPU
  {
  int prop1;
  int prop2;
  char* text; //point to a location on CPU 
      //this location hold the text say, "Hellow World"
  }
  
GPU
  {
  int prop1;
  int prop2;
  char* text; //point to a location on GPU
  }
\end{verbatim}
When doing copy, we need to do 2 copies
\begin{itemize}
  \item copy individual data elements
  
  \item copy the memory region referenced by \verb!*text!
\end{itemize}
\begin{verbatim}
 dataElem *elem; //data on HOST (CPU-side)

 dataElem *g_elem;
 char *g_text;
 int textlen = strlen(elem->text);
 
// Allocate storage for struct and text
 cudaMalloc(&g_elem, sizeof(dataElem));
 cudaMalloc(&g_text, textlen);
 

cudaMemcpy(g_elem, elem, sizeof(dataElem));
cudaMemcpy(g_text, elem->text, textlen);
 //copy the address of text on GPU to the data element
 // of the object on GPU
cudaMemcpy(&(g_elem->text), &g_text, sizeof(g_text));
 
 
kernel<<< .., ..>>> (g_elem);
\end{verbatim}


\subsection{-- Since Unified Memory: class must be managed}

Don't use pointer inside the data element, use object of class that is managed (see 3 steps below)
\begin{verbatim}
// Ideal C++ version of class
class dataElem {
 int prop1;
 int prop2;
 String text;
};

\end{verbatim}

A managed class is a class whose C++ objects are always allocated on managed heap

\textcolor{red}{Step 1}: write a Managed class as a base for any Unified-Memory-friendly class
\begin{verbatim}
class Managed {
 void *operator new(size_t len) {
   void *ptr;
   cudaMallocManaged(&ptr, len);
   return ptr;
 }
 void operator delete(void *ptr) {
 cudaFree(ptr);
 }
};
\end{verbatim}

\textcolor{red}{Step 2}: Then make the class String being managed, and implement copy constructor
\begin{verbatim}
class String : public Managed {
 int length;
 char *data;
 
 // Copy constructor using new allocates CPU-only data
 String (const String &s) {
 length = s.length;
 data = new char[length+1]; //which becomes
 /*
   // Unified memory copy constructor allows pass-by-value
   cudaMallocManaged(&data, length+1);
 */
 strcpy(data, s.data);
 }
};
\end{verbatim}

\textcolor{red}{CPU/GPU class sharing is restricted to POD-classes only (i.e. no
virtual functions)}.

\textcolor{red}{Step 3}: Make the class dataElem also managed
\begin{verbatim}
// Note “Managed” on this class, too.
// C++ now handles our deep copies
class dataElem : public Managed {
 int prop1;
 int prop2;
 String text;
};
\end{verbatim}
Copy constructors from CPU create GPU-usable objects.

\textcolor{red}{Step 4}: For an array of class objects
\begin{verbatim}
template <class T>
class Array : public Managed {
 size_t n;
 T* data;
public:
 Array (const Array &a) {
   n = a.n;
   cudaMallocManaged(&data, n);
   memcpy(data, a.data, n);
 }

 // Also have to implement operator[], for example
 // ...
};


int main(void) {
 Array *a = new Array; //on Unified Memory an array object
 ...
 // pass data to kernel by reference
 kernel_by_ref<<<1,1>>>(*a);
 // pass data to kernel by value -- this will create a copy
 kernel_by_val<<<1,1>>>(*a);
}


// Pass-by-reference version
__global__ void kernel_by_ref(dataElem &data) { ... }

// Pass-by-value version [CUDA knows how to copy]
__global__ void kernel_by_val(dataElem data) { ... }


\end{verbatim}




\url{https://devtalk.nvidia.com/default/topic/1026116/cuda-programming-and-performance/copying-objects-to-device-with-virtual-functions/}

\subsection{array class in CUDA Unified Memory}


NOTE: The use of \verb!_start! and \verb!_end!. 

\url{https://www.quantstart.com/articles/dev_array_A_Useful_Array_Class_for_CUDA}

\url{https://stackoverflow.com/questions/10375680/using-stdvector-in-cuda-device-code}

\url{https://github.com/NVIDIA-developer-blog/code-samples/blob/master/posts/unified-memory/dataElem_um_c++_2.cu}

NOTE: Any typename T passed to LocalVector should be either intrinsic type (float, double, int) 
or a user-defined class inherited from \verb!Managed! class (see definition below)

\begin{lstlisting}
template<typename T>
class LocalVector
{
private:
    T* m_begin;
    T* m_end;

    size_t capacity;
    size_t length;
    __device__ void expand() {
        capacity *= 2;
        size_t tempLength = (m_end - m_begin);
        T* tempBegin = new T[capacity];

        memcpy(tempBegin, m_begin, tempLength * sizeof(T));
        delete[] m_begin;
        m_begin = tempBegin;
        m_end = m_begin + tempLength;
        length = static_cast<size_t>(m_end - m_begin);
    }
public:
    __device__  explicit LocalVector() : length(0), capacity(16) {
        m_begin = new T[capacity];
        m_end = m_begin;
    }
    __device__ T& operator[] (unsigned int index) {
        return *(m_begin + index);//*(begin+index)
    }
    __device__ T* begin() {
        return m_begin;
    }
    __device__ T* end() {
        return m_end;
    }
    __device__ ~LocalVector()
    {
        delete[] m_begin;
        m_begin = nullptr;
    }

    __device__ void add(T t) {

        if ((m_end - m_begin) >= capacity) {
            expand();
        }

        new (m_end) T(t);
        m_end++;
        length++;
    }
    __device__ T pop() {
        T endElement = (*m_end);
        delete m_end;
        m_end--;
        return endElement;
    }

    __device__ size_t getSize() {
        return length;
    }
};
\end{lstlisting}

Example: Managed class
\begin{lstlisting}
// Managed Base Class -- inherit from this to automatically 
// allocate objects in Unified Memory
class Managed 
{
public:
  void *operator new(size_t len) {
    void *ptr;
    cudaMallocManaged(&ptr, len);
    cudaDeviceSynchronize();
    return ptr;
  }

  void operator delete(void *ptr) {
    cudaDeviceSynchronize();
    cudaFree(ptr);
  }
};

\end{lstlisting}

\subsection{using unique\_ptr for array}

\url{https://stackoverflow.com/questions/16711697/is-there-any-use-for-unique-ptr-with-array}

\url{https://github.com/eyalroz/cuda-api-wrappers/blob/master/src/cuda/api/unique_ptr.hpp}

\subsection{Unified Memory whose data element is a pointer (which can point to host-allocated memory?)}

Suppose you have allocated some memory on host (using standard new) and trying
to reuse them as unified memory pointer.
\begin{verbatim}

Variable* v = new LifeDataCollector; // standard ::new

ShallowArray<Variable*> _variableList ; // whose internal data 'Variable** _data' 
             // is on UM, i.e. cudaMallocManaged(&_data, sizeof(Variable*) * num_elements);
             
_variableList.push_back(v) ;   // can crash 
\end{verbatim}

If that is the case, I think you’ll have to copy data from the host memory to
unified memory.
 
IMPORTANT:  host memory and managed memory are totally of different kinds and
you cannot mix up using their pointers.
Either you can use unified/managed memory from the beginning or you can copy
host memory to unified memory when needed. You can simply use memcpy to make the
transfer happen.

IMPORTANT: On an IBM Power9 platform, your host allocated data can still be
accessed from device code, however there is currently no corresponding method on
x86 platforms. \url{https://docs.nvidia.com/cuda/cuda-c-programming-guide/index.html#um-system-allocator}


\begin{verbatim}
cudaMallocManaged(&new_pointer, size); memcpy(new_pointer, old_pointer, size); free(old_pointer);

\end{verbatim}
\url{https://stackoverflow.com/questions/51750543/cudamallocmanaged-for-host-initiated-variable}


\url{https://www.quora.com/In-CUDA-how-can-a-host-pointer-be-added-to-Unified-Memory-Management-similar-to-using-cudaMallocManaged-but-using-host-array}


\subsection{smart pointer in CUDA}

Sect.\ref{sec:smart_pointer}


\url{https://ernestyalumni.wordpress.com/2017/09/28/bringing-cuda-into-the-year-2011-c11-smart-pointers-with-cuda-cub-nccl-streams-and-cuda-unified-memory-management-with-cub-and-cublas/}


\subsection{tree structure on Unified Memory}
\label{sec:UM_number-of-cudaMallocManaged-call}


I'm trying to build a tree structure that I want to use in a Cuda kernel. For
that I allocate memory with cudaMallocManaged() so that Cuda copies the
structure to the GPU when needed. But no matter the size of the structure, I get
an "out of memory" error at a maximum number of calls

Te key to understanding this is the number of allocations, at the failure point,
rather than their size. 65451 is suspiciously close to 65535 (i.e. 2^16).
There is some sort of accidental or deliberate limit on the total number of
memory managed memory allocations to 65535.

\begin{itemize}
  \item GeForce GTX 980 with Cuda 7.5 on a Ubuntu 14.04 desktop.

about the 65407th memory allocation, even though plenty of memory space is still
available on both CPU and GPU memory.
\url{https://stackoverflow.com/questions/38078737/cudamallocmanaged-returns-out-of-memory-despite-enough-free-space?noredirect=1&lq=1}

The overall dynamic allocation limit for this is approximately 65535.
\url{https://devtalk.nvidia.com/default/topic/950272/cudamallocmanaged-can-not-exceed-more-than-65410-iterartions-/}
  
  \item The behavior appears to be different in CUDA 8RC (possibly fixed)
  
  on a CUDA 8.0RC system, CentOS7, Tesla K20X (6GB). It appeared to run successfully through 1000000 iterations/allocations. The final printout line: 1716.62MB/ 5700.38MB (Free/ Total), #Nodes: 999999
  
  \item 
\end{itemize}

\begin{lstlisting}
#include <stdio.h>
#include <cuda_runtime.h>

typedef struct Tree {
    u_int64_t       foo;
    struct Tree     *left_child;
    struct Tree     *right_child;
} Tree;

void printMemInfo(size_t counter) {
  size_t freemem;
  size_t totalmem;
  cudaMemGetInfo(&freemem, &totalmem);
  printf("%4.2fMB/ %4.2fMB (Free/ Total), #Nodes: %lu\n",
    freemem / (1024 * 1024.0),
    totalmem / (1024 * 1024.0),
    counter);
}

int main (int argc, char *argv[]) {
  cudaSetDevice(0);
  size_t numnodes = 1000000;
  Tree *node[numnodes];

  for(size_t i = 0; i < numnodes; i++) {
    printMemInfo(i);
    cudaError_t error = cudaMallocManaged( (void **) &node[i], sizeof(Tree) );
    printCudaError(error);
  }

  cudaDeviceReset();
  return 0;
}
\end{lstlisting}

\textcolor{red}{Example 02}:
an array of type Bucket,which has a nested array ObjBox

There are totally N(70000) Bucket in the array, M(1000) ObjBox in each Bucket.
\url{https://stackoverflow.com/questions/34850411/allocate-unified-memory-in-my-program-aftering-running-it-throws-cuda-errorou}
\begin{lstlisting}
#define N 70000
#define M 1000

class ObjBox
{public:

    int oid; float x; float y; float ts
};

class Bucket
{public:

    int bid; int nxt; 
    ObjBox *arr_obj; 
    int nO;
}

int main()
{

   Bucket *arr_bkt;

   cudaMallocManaged(&arr_bkt, N * sizeof(Bucket));

   for (int i = 0; i < N; i++)

   {

       arr_bkt[i].bid = i; 

       arr_bkt[i].nxt = -1;

       arr_bkt[i].nO = 0;

       cudaError_t r = cudaMallocManaged(&(arr_bkt[i].arr_obj), M * sizeof(ObjBox));

       if (r != cudaSuccess)

       {
           printf("CUDA Error on %s\n", cudaGetErrorString(r));
           exit(0);
       }

       for (int j = 0; j < M; j++)

       {

           arr_bkt[i].arr_obj[j].oid = -1;

           arr_bkt[i].arr_obj[j].x = -1;

           arr_bkt[i].arr_obj[j].y = -1;

           arr_bkt[i].arr_obj[j].ts = -1;

        }

   }

   cout << "Bucket Array Initial Completed..." << endl;

   cudaFree(arr_bkt);

   return 0;

}
\end{lstlisting}

\subsection{queue}

Compile
\begin{verbatim}
nvcc -std=c++11 offsetof_error.cpp -o app -Xcudafe "--diag_suppress=1427"

// to get various error/warning ID
cudafe --display_error_number
\end{verbatim}

\begin{lstlisting}
#include <iostream>
#include <typeinfo>
#include <type_traits>
#include <cstddef>

using namespace std;

template <typename T> struct queue_t
{
    T *head;
    T *tail;
};

struct data
{
    int foo;
};

template<typename T> class myclass
{
    static_assert(std::is_pod<T>::value,            "Queue Payload must be POD!");
    static_assert(std::is_pod< queue_t<T> >::value, "Queue Payload must be POD!");

public:
    myclass()
    {
        cout << typeid(T).name() << " is pod? " << is_pod< queue_t<T> >::value << endl;
        cout << "Offset of head is: " << offsetof(queue_t<T>, head) << endl;
    }
};

int main()
{
    myclass<data> test;
    return 0;
}
\end{lstlisting}
\url{https://devtalk.nvidia.com/default/topic/856171/unnecessary-compiler-warning-with-nvcc-/}


\subsection{Class Template}

\url{http://pleiades.ucsc.edu/doc/cuda/6.0/cuda-samples/index.html}

\subsection{Virtual Base Class}

This (Sect.\ref{sec:virtual_base_class-C++}) is supported since CUDA 4.0, Fermi
Card (Sect.\ref{sec:CUDA_4.0}).


\subsection{Pure virtual function in CUDA}

Pure virtual functions (Sect.\ref{sec:OO_pure-virtual-function}) is now
supported on CUDA 4.0 and fermi card (Sect.\ref{sec:CUDA_4.0}).


When a function in a derived class overrides a virtual function in a base class,
the execution space specifiers (i.e., \verb!__host__, __device__!) on the overridden
and overriding functions must match.

\url{https://docs.nvidia.com/cuda/cuda-c-programming-guide/index.html#virtual-functions}


\subsection{An object from a class having virtual function}


IMPORTANT: If creating these objects on the host and copying them to the device
\begin{verbatim}
"It is not allowed to pass as an argument 
   to a __global__ function an object of a class with virtual functions. "


The reason is that if you instantiate the object on the host, then the virtual
function table gets populated with host pointers. When you copy this object to
the device, these host-pointers become meaningless.

\end{verbatim}

SOLUTION: all of my virtual code to run only on the device and made an
initialization kernel and then everything worked.

EXPLAIN: If you create the objects on the device (you can still configure and
populate the objects with data passed from the host) then virtual functions (and
polymorphism) should not be a problem.
\url{https://devtalk.nvidia.com/default/topic/825046/can-cuda-properly-handle-pure-virtual-classes-/?offset=4}

REFERENCE: \url{https://docs.nvidia.com/cuda/cuda-c-programming-guide/index.html#virtual-functions}


\subsection{Complex structure whose data element is a pointer}
\label{sec:complex-datastructure-data-element-is-pointer}


Consider you have data structure
\begin{verbatim}

struct dataElem
{
   int prop1;
   int prop2;
   char* name;

}
\end{verbatim}

Now, to use this in CPU and then copy to GPU, we need deep-copies. Here, we need two copies.

To use this structure on the device, we have to copy the struct itself with its
data members, and then copy all data that the struct points to, and then update
all the pointers in copy of the struct.

\begin{lstlisting}

void launch(dataElem *elem) {
  dataElem *d_elem;
 
  char *d_name;

  int namelen = strlen(elem->name) + 1;

  // Allocate storage for struct and 
  //   storage to be pointed by the data member as pointer, i.e. name
  cudaMalloc(&d_elem, sizeof(dataElem));
  cudaMalloc(&d_name, namelen);

  // Copy up each piece separately, including new “name” pointer value
  
  // first copy for structure
  cudaMemcpy(d_elem, elem, sizeof(dataElem), cudaMemcpyHostToDevice);
  
  // second copy for the internal data as pointer
  // (we may need more if there is more pointer data memebers)
  cudaMemcpy(d_name, elem->name, namelen, cudaMemcpyHostToDevice);
  
  // finally, making d_elem->name point to 
  cudaMemcpy(&(d_elem->name), &d_name, sizeof(char*), cudaMemcpyHostToDevice);

  // Finally we can launch our kernel, but CPU & GPU use different copies of “elem”
  Kernel<<< ... >>>(d_elem);
}
\end{lstlisting}


With Unified Memory (Sect.\ref{sec:Unified-Memory-CUDA6.0}), the shallow copies
(or deep copies) is managed automatically. Allocating our dataElem structure in
Unified Memory eliminates all the excess setup code, leaving us with just the
kernel launch, which operates on the same pointer as the host code. That’s a big
improvement!

HOW? We introduce a Managed basedclass (Sect.\ref{sec:UnifiedMemory-Managed-baseclass})
\begin{lstlisting}
// Note “managed” on this class, too.
// C++ now handles our deep copies
class dataElem : public Managed {
public:
  int prop1;
  int prop2;
  String name;  // NOTE we don't use pointer
};

// Deriving from “Managed” allows pass-by-reference
class String : public Managed {
  int length;
  char *data;

public:
  // Unified memory copy constructor allows pass-by-value
  String (const String &s) {
    length = s.length;
    cudaMallocManaged(&data, length);
    memcpy(data, s.data, length);
  }

  // ...
};
\end{lstlisting}

\begin{verbatim}
dataElem* elem;

cudaMallocManaged( &elem, sizeof(dataElem), );

void launch(dataElem *elem) {
  kernel<<< ... >>>(elem);
  cudaDeviceSynchronize();
  
}
\end{verbatim}

\url{https://devblogs.nvidia.com/unified-memory-in-cuda-6/}

\subsection{CPU/GPU shared linked list}

Linked lists are a very common data structure, but because they are essentially
nested data structures made up of pointers, passing them between memory spaces
is very complex.
\begin{lstlisting}
struct dataElem{
   int internal_data;

   dataElem* next;
}


dataElem*  linked_list;
\end{lstlisting}

Without Unified Memory, sharing a linked list between the CPU and the GPU is
unmanageable.
Before Unified Memory, the only option is to allocate the list in Zero-Copy
memory (pinned host memory), which means that GPU accesses are limited to
PCI-express performance.

By allocating linked list data in Unified Memory, device code can follow
pointers normally on the GPU with the full performance of device memory. The
program can maintain a single linked list, and list elements can be added and
removed from either the host or the device.
 
\begin{lstlisting}
// Note “managed” on this class, too.
// C++ now handles our deep copies
class dataElem : public Managed {
public:
  int prop1;
  int prop2;
  String name;
};
\end{lstlisting}
 
\subsection{Deep copy (copy constructor) with Unified Memory using a Managed base-class}
\label{sec:UnifiedMemory-Managed-baseclass}

C++ simplifies the deep copy problem by using classes with copy constructors
(Sect.\ref{sec:copy-constructor}).
A copy constructor is a function that knows how to create an object of a class,
allocate space for its members, and copy their values from another object.

C++ also allows the \verb!new! and \verb!delete! memory management operators to be overloaded.
 
We can create a base class, which we’ll call {\it Managed}, which uses
cudaMallocManaged() inside the overloaded new operator, as in the following
code.

\begin{lstlisting}

class Managed {
public:
  void *operator new(size_t len) {
    void *ptr;
    cudaMallocManaged(&ptr, len);
    cudaDeviceSynchronize();
    return ptr;
  }

  void operator delete(void *ptr) {
    cudaDeviceSynchronize();
    cudaFree(ptr);
  }
};
\end{lstlisting}

Now, we have our class inherit from this class

\begin{lstlisting}
// Deriving from “Managed” allows pass-by-reference
class String : public Managed {
  int length;
  char *data;

public:
  // Unified memory copy constructor allows pass-by-value
  String (const String &s) {
    length = s.length;
    cudaMallocManaged(&data, length);
    memcpy(data, s.data, length);
  }

  // ...
};
\end{lstlisting}
 

\subsection{Thrust with managed memory}
\label{sec:thrust-managed-memory}

\begin{lstlisting}
#include <iostream>
#include <cmath>
#include <thrust/reduce.h>
#include <thrust/system/cuda/execution_policy.h>
#include <thrust/system/omp/execution_policy.h>
 
const int ARRAY_SIZE = 1000;
 
int main(int argc, char **argv) {
    double* mA;
    cudaMallocManaged(&mA, ARRAY_SIZE * sizeof(double));
    
    thrust::sequence(mA, mA + ARRAY_SIZE, 1);
    
    double maximumGPU = thrust::reduce(thrust::cuda::par, mA, mA + ARRAY_SIZE, 0.0,      
                                     thrust::maximum<double>());
    cudaDeviceSynchronize();
    double maximumCPU = thrust::reduce(thrust::omp::par, mA, mA + ARRAY_SIZE, 0.0,    
                                       thrust::maximum<double>());
    std::cout << "GPU reduce: “ << (std::fabs(maximumGPU ‐ ARRAY_SIZE) < 1e‐10 ? "Passed" : "Failed");  
    std::cout << "CPU reduce: “ << (std::fabs(maximumCPU ‐ ARRAY_SIZE) < 1e‐10 ? "Passed" : "Failed"); 
    cudaFree(mA);
    return 0;
}
\end{lstlisting}

Note that using Thrust with managed memory requires the latest development
version Thrust v1.8 (the CUDA Toolkit only provides Thrust v1.7).


\url{http://www.drdobbs.com/parallel/unified-memory-in-cuda-6-a-brief-overvie/240169095?pgno=2}

\subsection{Sparse linear matrix}

SPIKE::GPU, a library for the GPU solution of mid-size sparse linear systems, and use it to assess the performance of managed memory in CUDA 6.


SPIKE::GPU is an open source sparse linear solver for systems of medium size;
that is, up to approximately 0.5 million unknowns. The solution strategy has
three steps. In step 1, a sparse matrix is reordered; that is, entries in A are
moved around by shifting the order of the rows and columns in the matrix to
accomplish two things (Figure 3):

\url{http://www.drdobbs.com/parallel/unified-memory-in-cuda-6-a-brief-overvie/240169095?pgno=2}

\url{https://github.com/spikegpu/SaPLibrary}


\section{Organizing CUDA C++ codes}


Sect.\ref{sec:modules-in-C++} discuss how to organize C++ codes into compilation
units (.cpp/.h).

\textcolor{red}{Consider the CUDA C++ example}:  a \verb!particle! class and a three-dimensional
vector class, \verb!v3!, that it uses.

The class contains a number of data member, as instances of another class.
\begin{verbatim}
#include <v3.h>

class particle
{
private:
    v3 position;
    v3 velocity;
    v3 totalDistance;

public:
    particle();
    __host__ __device__ void advance(float dist);
    const v3& getTotalDistance() const;
};
\end{verbatim}

We’ll use a CUDA C++ kernel in which each thread calls
\verb!particle::advance()! on a particle, which inturns call other CUDA C++
kernels, e.g. those from V3 class.

\subsection{particle class}


Sect.\ref{sec:CUDA_class} describes how to write a class that is both CPU and GPU friendly.

\begin{verbatim}
#include <particle.h>

particle::particle() : position(), velocity(), totalDistance(0,0,0) {}

__device__ __host__ 
void particle::advance(float d)
{
    velocity.normalize();
    float dx = d * velocity.x;
    position.x += dx;
    totalDistance.x += dx;
    float dy = d * velocity.y;
    position.y += dy;
    totalDistance.y += dy;
    float dz = d * velocity.z;
    position.z += dz;
    totalDistance.z += dz;
    velocity.scramble();
}

const v3& particle::getTotalDistance() const
{
    return totalDistance; 
}
\end{verbatim}


\subsection{Array of class objects}

Now, we have an array of class objects which is the object-oriented way of programming. 

How can we make it running on GPU?

At the CompCategory level, we need to define a \verb!__global__! method to do it
\begin{verbatim}
#include <particle.h>
#include <stdlib.h>
#include <stdio.h>

__global__ 
void advanceParticles(float dt, particle * pArray, int nParticles)
{
    int idx = threadIdx.x + blockIdx.x*blockDim.x;
    if(idx < nParticles) { pArray[idx].advance(dt); } 
} 


int main(int argc, char ** argv) 
{     
    int n = 1000000;     
    if(argc > 1) { n = atoi(argv[1]);}     // Number of particles
    if(argc > 2) { srand(atoi(argv[2])); } // Random seed

    particle * pArray = new particle[n];
    particle * devPArray = NULL;
    cudaMalloc(&devPArray, n*sizeof(particle));
    cudaMemcpy(devPArray, pArray, n*sizeof(particle), cudaMemcpyHostToDevice);
    for(int i=0; i<100; i++)
    {   // Random distance each step
        float dt = (float)rand()/(float) RAND_MAX;
        advanceParticles<<< 1 +  n/256, 256>>>(dt, devPArray, n);
        cudaDeviceSynchronize();
    }

    cudaMemcpy(pArray, devPArray, n*sizeof(particle), cudaMemcpyDeviceToHost);
    v3 totalDistance(0,0,0);
    v3 temp;
    for(int i=0; i<n; i++)
    {
        temp = pArray[i].getTotalDistance();
        totalDistance.x += temp.x;
        totalDistance.y += temp.y;
        totalDistance.z += temp.z;
    }
    float avgX = totalDistance.x /(float)n;
    float avgY = totalDistance.y /(float)n;
    float avgZ = totalDistance.z /(float)n;
    float avgNorm = sqrt(avgX*avgX + avgY*avgY + avgZ*avgZ);
    printf("Moved %d particles 100 steps. Average distance traveled is |(%f, %f, %f)| = %f\n", 
                                          n, avgX, avgY, avgZ, avgNorm);
    return 0;
}
\end{verbatim}

\subsection{V3 class }


\begin{verbatim}
class v3
{
public:
    float x;
    float y;
    float z;

    v3();
    v3(float xIn, float yIn, float zIn);
    void randomize();
    __host__ __device__ void normalize();
    __host__ __device__ void scramble();
};
\end{verbatim}


\begin{verbatim}
#include <v3.h>
#include <math.h>

v3::v3() { randomize(); }

v3::v3(float xIn, float yIn, float zIn) : x(xIn), y(yIn), z(zIn) {}

void v3::randomize()
{
    x = (float)rand() / (float)RAND_MAX;
    y = (float)rand() / (float)RAND_MAX;
    z = (float)rand() / (float)RAND_MAX;
}

__host__ __device__ void v3::normalize()
{
    float t = sqrt(x*x + y*y + z*z);
    x /= t;
    y /= t;
    z /= t;
}

__host__ __device__ void v3::scramble()
{
    float tx = 0.317f*(x + 1.0) + y + z * x * x + y + z;
    float ty = 0.619f*(y + 1.0) + y * y + x * y * z + y + x;
    float tz = 0.124f*(z + 1.0) + z * y + x * y * z + y + x;
    x = tx;
    y = ty;
    z = tz;
}
\end{verbatim}


\subsection{C++ CUDA in CUDA 4.0}


Before CUDA 5.0, if a programmer wanted to call particle::advance() from a CUDA
kernel launched in main.cpp, the compiler required the main.cpp compilation unit
to include the implementation of particle::advance() as well any subroutines it
calls (v3::normalize() and v3::scramble() in this case).

In complex C++ applications, the call chain may go deeper than the two-levels that our example illustrates.

Without device object linking, the developer may need to deviate from the
conventional application structure to accommodate this compiler requirement.
Such changes are difficult for existing applications in which changing the
structure is invasive and/or undesirable.


\subsection{C++ CUDA in CUDA 5.0}


From CUDA 5.0, the compiler can generate device code for all functions in a .cpp
file, store it in a single .o file, and then link device code from multiple .o files
together in the same way that we are used to linking CPU code.

From existing C++ code, the only required change in v3.h, v3.cpp, particle.h,
and particle.cpp is to add
\begin{verbatim}
__host__  __device__ 
\end{verbatim}
decorators to member functions that device code calls.

The implementations are otherwise completely unchanged from their CPU-only version.


The \verb!__host__ __device__! decorations indicate to nvcc to compile these routines into both CPU code and device-callable GPU code. 
\begin{itemize}
  \item   Using \verb!__host__! alone tells the compiler to generate only a CPU version of this routine
  
  This usage is unnecessary, as this is the default behavior. 
  
  \item   Using \verb!__device__! alone tells the compiler to generate only a GPU version of this routine

This is useful if you know this routine will never be needed by the host, or if
you want to implement your function using operations specific to the GPU, such
as fast math or texture unit operations
\end{itemize}


If you call a \verb!__host__! function from the device or a \verb!__device__! function from the host, the compiler will report an error.


\subsection{Building C++ CUDA-friendly code}

NOTE: Instead of using \verb!g++!, we use 
\verb!nvcc -x cu -arch=sm_20! 

Example:
\begin{verbatim}
objects = main.o particle.o v3.o

all: $(objects)
    nvcc -arch=sm_20 $(objects) -o app

%.o: %.cpp
    nvcc -x cu -arch=sm_20 -I. -dc $< -o $@

clean:
    rm -f *.o app
\end{verbatim}


\subsection{MPI and CUDA code in same file}

\begin{verbatim}
nvcc -I/usr/mpi/gcc/openmpi-1.4.6/include -L/usr/mpi/gcc/openmpi-1.4.6/lib64 -lmpi \
    spaghetti.cu -o program
\end{verbatim}

\subsection{MPI and CUDA code are in separated files}

Another cleaner option is to have MPI and CUDA code separate in two files:
main.c and multiply.cu respectively. These two files can be compiled using
mpicc, and nvcc respectively into object files (.o) and combined into a single
executable file using mpicc.

\begin{verbatim}
module load openmpi cuda #(optional) load modules on your node
mpicc -c main.c -o main.o
nvcc -arch=sm_20 -c multiply.cu -o multiply.o
mpicc main.o multiply.o -lcudart -L/apps/CUDA/cuda-5.0/lib64/ -o program
\end{verbatim}
using mpicc, meaning that you have to link to your CUDA library

\url{https://anhnguyen.me/2013/12/how-to-mix-mpi-and-cuda-in-a-single-program/}

NOTE: request 2 processes and 2 GPUs
\begin{verbatim}
#PBS -l nodes=2:ppn=2:gpus=2
mpiexec -np 2 ./program
\end{verbatim}

\section{Error: dynamic initialization not supported}
\label{sec:constant_memory_C++}

\begin{verbatim}
error : dynamic initialization is not supported for __device__, __constant__ and
__shared__ variables.
\end{verbatim}
\url{http://stackoverflow.com/questions/27041820/why-am-i-getting-dynamic-initialization-not-supported-for-device-constant}

\begin{verbatim}
#include "../Params.hpp"

__constant__ Params cparams;


template <typename T>
class Vec2
{
public:
   Vec2(){ x = 0.0; y = 0.0;}
   T x, y;
};

typedef Vec2<float> vectorTypef;
typedef struct Params
{
   vectorTypef a;
} Params;
\end{verbatim}

A data member of the class Params is an object of \verb!Vec2! class. This
\verb!Vec2! class has a default constructor, which assign data to the data
members. This method of assigning data to a \verb!__constant__! region is not
legal either in device code or host code.

In other words, there are only 1 way to assign value to a constant memory,
and cannot be done via the constructor.
\begin{itemize}
  \item assign in the host code, \verb!cudaMemcpyToSymbol()! for CUDA C or
  simple assignment in CUDA Fortran.
\end{itemize}
I would recommend that you assign these in your host code explicitly, rather
than via a constructor.

So, one possible approach to fix this would be to change your default
constructor to an empty one:
\begin{verbatim}
public:
   __host__ __device__ Vec2(){ }; // change this line
   T x, y;
\end{verbatim}
and then we use
\begin{verbatim}
Params hparams;
hparams.a.x = 0.0;
hparams.a.y = 0.0;
cudaMemcpyToSymbol(cparams, &hparams, sizeof(Params));
\end{verbatim}
% \section{cuDPP}
% \label{sec:cudpp}
