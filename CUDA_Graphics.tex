\chapter{CUDA Graphics (OpenGL/DirectX)}
\label{chap:CUDA_OpenGL.DirectX}

A CUDA application can interact (write/read) with data from OpenGL/Direct3D
APIs by mapping the resource from OpenGL/Direct3D to the address space of CUDA.
Currently, this feature is only available in CUDA C/C++. 

At the beginning of the program, the CUDA device need to be enabled with
OpenGL/DirectX interoperability using
\begin{verbatim}
cudaGLSetGLDevice(dev_ID)
\end{verbatim}
\textcolor{red}{IMPORTANT: cudaGLSetGLDevice() and cudaSetDevice() are mutually
exclusive. So, we only use one of them}. We use cudaSetDevice()   in a
pure computational CUDA application.

{\bf What can be registerred and mapped?}: OpenGL buffers, texture, and
render-buffer objects.

The orders:
\begin{enumerate}
  \item register
  \item map
  \item use in CUDA
  \item unmap
  \item repeat 2
  \item unregister
\end{enumerate}

To register the resource with CUDA
\begin{verbatim}
 // register a buffer
 cudaGraphicsGLRegisterBuffer()
 
 // register a texture or render-buffer object
 cudaGraphicsGLRegisterImage()
\end{verbatim}
\textcolor{red}{Register a resource is potentially high-overhead, so you're
recommended to register once per resource}. To unregister, use 
\begin{verbatim}
cudaGraphicsUnregisterResource()
\end{verbatim}
\begin{enumerate}
  \item The function cudaGraphicsGLRegisterBuffer() returns a pointer to a CUDA graphics
object of type
\begin{verbatim}
struct cudaGraphicsResource
\end{verbatim}
and can be read/written by kernels or via \verb!cudaMemcpy()! function.

The function also supports all texture formats with 1, 2 or 4 components with
the following internal types
\begin{verbatim}
 // float
GL_RGBA_FLOAT32
 // normalized integer
 GL_RGBA8, GL_INTENSITY16
 // unnormalized integer
 GL_RGBA8UI
\end{verbatim}

\textcolor{red}{IMPORTANT: unnormalized integer formats require OpenGL 3.0; and
thus written by shaders (CUDA cores), not fixed function pipeline}. 


 \item The second function returns as a CUDA array which can be read/written
 via \verb!cudaMemcpy2D()! call. To read from a kernel, it need to be bind to
 a texture or a surface reference. When the resource is registered with
 \verb!cudaGraphicsRegisterFlagsSurfaceLoadStore! flag (using 
 cudaGraphicsResourceSetMapFlags()), it can be written via surface write 
 functions.
\end{enumerate}


Once it's registerred, it can be mapped or unmapped as many time as necessary
using 
\begin{verbatim}
cudaGraphicsMapResources()
cudaGraphicsUnmapResources()
\end{verbatim}
Then, it can be used from the CUDA kernel using the memory address returned by
\begin{verbatim}
 // for buffers
cudaGraphicsResourceGetMappedPointer()

 // for CUDA arrays
 cudaGraphicsSubResourceGetMappedArray() 
\end{verbatim}
When we know the resource is being used for read-only or write-only, we can 
optimize resource management, by using  
\begin{verbatim}
cudaGraphicsResourceSetMapFlags() 
\end{verbatim}
to specify usage hints. 

\textcolor{red}{While the resource is mapped, OpenGL/Direct3D APIs can't use
them. Otherwise, undefined results are returned}. 


\section{OpenGL vs. DirectX interoperability}
\label{sec:opengl-vs.-directx}

\verb!cudaGLMapBufferObject()!

\verb!cudaD3D9MapVertexBuffer()!



\section{Geforce}

  \url{http://www.geforce.com/whats-new/articles/geforce-experience-2-1-2-released}
\begin{enumerate}
  \item Dynamic Super Resolution (DSR)
  \item 4K ShadowPlay recording

\end{enumerate}
\subsection{GameStream}

\begin{enumerate}
  \item Notebook GameStream
  \item Remotely wake-on-LAN (from sleep mode)
  \item Remote Streaming: stream PC games to SHIELD from outside your home
\end{enumerate}



\subsection{ShadowPlay}

\subsection{SHIELD game controller}

\url{http://shield.nvidia.com/wireless-game-controller/}