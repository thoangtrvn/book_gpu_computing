%%
%% UnitTest.tex
%% Login : <hoang-trong@hoang-trong-laptop>
%% Started on  Thu Jul 23 13:02:24 2009 Hoang-Trong Minh Tuan
%% $Id$
%% 
%% Copyright (C) 2009 Hoang-Trong Minh Tuan
%%

\chapter{Unit Testing}
\label{chap:unit-testing}

There are two major choices for Fortran Unit Testing\footnote{\url{http://fortranwiki.org/fortran/show/Unit+testing+frameworks}}\footnote{\url{http://www.sesp.cse.clrc.ac.uk/Publications/tools_report_2006/node11.html}}:
\begin{itemize}
\item Funit\footnote{\url{http://nasarb.rubyforge.org/}}
\item FRUIT\footnote{\url{http://fortranwiki.org/fortran/show/Fortran+Unit+Test+Framework}}
\end{itemize}

\section{FUnit}
\label{sec:funit}

FUnit stands for Fortran Unit testing. For detail, read
\url{http://jblevins.org/computing/fortran/funit/}
\begin{enumerate}
\item Download the .gem file

\item Add FC environment variable to the correct Fortran compiler,
  e.g. in the ~/.bashrc file
\begin{verbatim}
export FC="pgf95"
\end{verbatim}

\item Install
\begin{verbatim}
sudo gem install funit-0.10.2.gem
sudo gem install rails
\end{verbatim}
\end{enumerate}

\subsection{Limitation}
\label{sec:limitation}

\begin{enumerate}
\item Function/subroutines to be tested need to be in a module

\item The module name (e.g. calculator.f95) and the test file has the
  same name, except the extension of the test file is .fun
  (e.g. calculator.fun)

\item Each test file can contain multiple tests, as well as
  {\it setup} and {\it teardown} command that are executed once for
  each test.

\item There are 6 types of assertions:
\begin{verbatim}
assert_true(expr)

assert_false(expr)

assert_equal(a, b)

assert_real_equal(a, b)

assert_equal_with(a, b, tol)

assert_array_equal(a,b) ! no space after comma
\end{verbatim}

\item FUnit can tell (1) which test failed, (2) which assertion
  failed, (3) which line of the offending assertion is on.

\item The test have to be aligned at the first column.
\end{enumerate}

\subsection{Write test codes}
\label{sec:write-test-codes}

Module need to be tested is {\it circle\_class.f95}
\begin{lstlisting}
module circle_class
  implicit none
  private
  public :: circle, circle_area

  type circle
    real :: radius
  end type circle
contains

  function circle_area(this) result(area)
     type(circle), intent(in) :: this
     ...
  end function circle_area

end module circle_class
\end{lstlisting}

\begin{enumerate}
\item Create a .fun file with the same name of the module  whose
  subroutines/functions need to be tested.

\item The test suite look likes
\begin{lstlisting}
test_suite circle_class
  ! declare global variables
  real :: c
  real, parameter :: radius = 1.5d0

setup
  ! code that you want to run before each test

  c = circle(radius)
end setup

teardown
  ! code that run immediately after each test
end teardown

test test_name_here

end test

test radius_is_stored_properly
  assert_real_equal(radius, 1.5d0)
end test

end test_suite
\end{lstlisting}

\item Run the test
\begin{lstlisting}
funit circle_class

! funit module_name
\end{lstlisting}

\item Clean up everything
\begin{lstlisting}
funit --clean
\end{lstlisting}

\end{enumerate}

Now, testing Fortran code using FUnit is very easy with makefile.
\begin{lstlisting}
test:
     funit circle_class
     <add more command here>

clean: 
     -rm *.o *.mod
     funit --clean
\end{lstlisting}


\section{FRUIT}
\label{sec:fruit}

FRUIT is an acronym for FoRtran UnIT testing framework. It is written
in Fortran 90 and thus support full features of Fortran 90/95. FRUIT
use TDD (Test-Driven Development) framework. Here are necessary steps
to make it works for your project.

{\bf This needs to be done only once}
\begin{enumerate}

\item Make sure these tools are installed
\begin{verbatim}
sudo apt-get install ruby ruby-gems1.9 rake irb
\end{verbatim}

\item Download the latest version (here is 2.7), extract it and rename
  the folder to \verb!fruit!. 

\item Jump to the fruit folder
\begin{verbatim}
cd fruit
cd fruit_processor_gem
gem install pkg/fruit_processor-2.7.gem 
\end{verbatim}

\end{enumerate}
{\bf IMPORTANT NOTES}: FRUIT is incompatible with
  \href{https://agora.cs.illinois.edu/display/cs427fa08/Unit+Testing+with+FRUIT}{gfortran
    in Macs}.

{\bf This should be done only once for each project}
\begin{enumerate}

\item copy the \verb!rakebase.rb! file to your project directory. Edit
  this file to make sure 
  \begin{enumerate}
  \item it contains the correct compiler
\begin{verbatim}
$compiler = 'pgf95'
\end{verbatim}
% \item adjust the file extension (default is .f90) and object file
%   extension (default is .o)
% \begin{verbatim}
%  SRC = FileList['*.f90'].sort
%  OBJ = SRC.ext('o')
%   ! adjust any other place with .f90 to the correct
%   ! extension, e.g. .f95
% \end{verbatim}
% \item You can also set some necessary directories
% \begin{verbatim}
%  $base_dir = FruitProcessor.new.base_dir if ! $base_dir
%  $build_dir = FruitProcessor.new.build_dir if ! $build_dir

%  $lib_bases = {} if !$lib_bases
%  $inc_dirs = [] if !$inc_dirs
% \end{verbatim}

\end{enumerate}

\item create the \verb!test! folder inside the project directory

\item adjust the makefile to have these information
\begin{verbatim}

\end{verbatim}

\item Copy the two files \verb!/src/fruit.f90, /src/fruit_util.f90! to
  your project directory.
% \item copy the two files /src/fruit.f90, /src/fruit\_util.f90 to your
%   \verb!test! folder.  It is recommended that you compile these two
%   file first
% \begin{verbatim}
% pgf95 -c fruit_util.f90
% pgf95 -c fruit.f90
% \end{verbatim}

%  If you're using f95, make sure to rename its extension to .f95.

% \item Create a rakefile in the project directory, suppose that
%   \verb!calculator! is the name of your project.

% \begin{verbatim}
% require 'rubygems'
% require 'fruit_processor'

% task :compile do
%   sh 'make'
% end

% task :test => :compile do
%   sh 'cd test;rake;./calculator_fruit_driver'
% end

% load "rake_base.rb"
% include RakeBase

% task :default => [:test]
% \end{verbatim}
\item Create a rakefile in your \verb!test! folder
\begin{verbatim}
! rakefile in test folder

\end{verbatim}
\end{enumerate}

{\bf This should be done for every module you want to test}: All test
modules should be created inside the \verb!test! folder.
\begin{enumerate}
\item Create the test file in the test folder: Suppose that the module
  file is \verb!calculator.f95!, then the test file should be
  \verb!calculator_test.f90! (NOTE: .f90 is required for FRUIT).
\begin{lstlisting}
module calculator
  implicit none
contains
  subroutine add (a, b, output)
    integer, intent (in) :: a, b
    integer, intent (out) :: output
    output=0
  end subroutine add
end module calculator
\end{lstlisting}
\begin{lstlisting}
-----------------------------------
module calculator_test
use fruit
 
contains
  ! setup_before_all
  ! setup = setup_before_each
  subroutine setup
     ...
  end subroutine setup
 
  ! teardown_before_all
  ! teardown = teardown_before_each
  subroutine teardown
    ...
  end subroutine teardown

  subroutine test_calculator_should_produce_4_when_2_and_2_are_inputs
    use calculator, only: add
    integer:: result
 
    call add (2,2,result)
    call assert_equals (4, result)
  end subroutine test_calculator_should_produce_4_when_2_and_2_are_inputs
end module calculator_test
\end{lstlisting}

\item Notice the structure of this test file: 
  \begin{enumerate}
  \item This test file is a module whose module name is the name of
    the module needed to test plus \verb!_test!,
    e.g. \verb!calculator_test!
  \item it call \verb!USE fruit!
  \item it needs a {\it setup} and a {\it teardown} subroutine to do
    necessary initiate and clean up work for each test
  \item then, you create each test procedure for each procedure in the
    module needed to test. The name of the test procedure is
    \verb!test_! plus the name of the source procedure.
  \end{enumerate}

% \item Finally, you need a driver program to call all these test
%   modules. 
% \begin{lstlisting}
% program fruit_driver
%   use fruit
%   use calculator_test
%   ! use all other test modules

%   call init_fruit

%   call test_calculator_should_produce_4_when_2_and_2_are_inputs
%   ! call all test test procedures

%   call fruit_summary
% end program fruit_driver
% \end{lstlisting}

% This driver program has 
% \begin{enumerate}
% \item USE fruit, USE the test module,
% \item call \verb!init_fruit!, 
% \item call all necessary test procedure in the test module, 
% \item call \verb!fruit_summary!.
% \end{enumerate}

%   {\bf IMPORTANT}: There are some steps can be done automatically.
%   This driver module can be generated automatically with a
%   {\it rakefile}

% \item
%   \textcolor{red}{From this step: you can choose an alternate option
%     ``using rakefile''}. Here, you create your own makefile,
%     e.g. \verb!test_makefile!

% \item Compile and run the compiled driver module.

% \item Whenever a new procedure is needed to test, you just repeat the
%   same steps.
\end{enumerate}

{\bf Summary}:
\begin{enumerate}
\item Write your code, create a makefile
% \item Create \verb!test! folder, put two fruit files inside the test
%   folder and compile them.
\item Create the \verb!test! folder, put two fruit files inside the
  project directory.
\item Adjust the content of the makefile.
\item Create the rakefile in the test directory and update them when
  new test module is available.
\item Create all test modules, given them the extension .f90, not .f95
  in the \verb!test! folder.
\item Compile your source files with \verb!make test!
\end{enumerate}

% Some definition in \verb!rake_base.rb! by default.
% \begin{verbatim}
% $base_dir=Dir.pwd
% $build_dir = "#{$base_dir}/build"

% !compile
%   sh "#{$compiler} -I#{$build_dir} -o #{$goal} #{OBJ} #{FruitProcessor.new.lib_name_flag($lib_bases, $build_dir)} #{FruitProcessor.new.lib_dir_flag($lib_bases, $build_dir)}"
% \end{verbatim}
% You should update the library information after the -I option by
% adding the new information in the rakefile
% \begin{verbatim}
% $build_dir = "FruitProcessor.new.base_dir"
% \end{verbatim}

{\bf rakefile}: This is the default rakefile in the test folder, you
can modify it if necessary. This rakefile to rake tool is similar to
the makefile to make tool.
\begin{verbatim}
require 'rubygems'
require 'fruit_processor'
 
# include this if there is generated code in this dir
FruitProcessor.new.pre_process
 
# name of the generated driver
$goal='calculator_fruit_driver'
 
$lib_bases = {'fruit' => FruitProcessor.new.build_dir,
  'calculator' => nil}
 
task :default => [$goal, :deploy, :spec]
 
task :gen do
  FruitProcessor.new.pre_process
end
 
task :spec do
  FruitProcessor.new.spec_report
end
 
#file ['calculator_test.o', 'module_b_test.o']  => '/users/chena/workspace/sourceforge_fruit/build/libfruit.a'
#file 'calculator_test.o'  => '/users/chena/workspace/sourceforge_fruit/build/libfruit.a'
 
load "#{FruitProcessor.new.base_dir}/rake_base.rb"
include RakeBase
\end{verbatim}
\begin{itemize}
\item You don't have to specify newly added module in the rakefile
  unless you want to specify the build order
\item By including RakeBase in the end, you include all the default
  build targets specified in the \verb!<fruit>/rake_base.rb!
\end{itemize} 

{\bf FRUIT provides these subroutines}: The obsolete naming
convention: \verb!assertName!, the new naming convention in FRUIT:
\verb!assert_name!.
\begin{enumerate}
\item assert\_equals(a, b)
\item assert\_not\_equals(a,b)
\item assert\_true(a)
\end{enumerate}

\section{Something about rake and rakefile}
\label{sec:someth-about-rakef}

Rake is a mnemonic for ``ruby make''. It is a tool that help you build
your source codes and is supposed to be better than make tool. Thus,
rakefile is like makefile; however, it contains all executable Ruby
code. Here are some basic components of a
rakefile\footnote{\url{http://rake.rubyforge.org/files/doc/rakefile_rdoc.html}}

It is more helpful if you are familiar with
\hyperref[chap:make-tool]{makefile}. 

\subsection{Task}
\label{sec:task}

This is the basic:
\begin{verbatim}
task: <name>
\end{verbatim}
with $<name>$ is any name.

A task can has prerequisites, given in a list enclosed in square
brackets following the name and an arrow \verb!=>!.
\begin{verbatim}
task: <name> => :prereq1
  ! more than one dependency
task: <name> => [ :prereq1, :prereq2 ]
\end{verbatim}

A task can has some actions, wrapped between the $do$ and $end$
keywords
\begin{verbatim}
task: compile => [ :prereq1, :prereq2 ] do 
   sh 'cd src; rake'
end

task: test => [ :prereq1, :prereq2 ] do 
   sh 'cd test; rake; ./calculator_fruit_driver'
end
\end{verbatim}

A default task to run when you invoke rake without any task name
\begin{verbatim}
task: default => [:test]
\end{verbatim}


%%% Local Variables: 
%%% mode: latex
%%% TeX-master: "gpucomputing"
%%% End: 
