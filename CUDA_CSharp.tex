\chapter{CUDA in C\#}
\label{chap:CUDA_Csharp}

First need to add Intellisense support for CUDA C in Visual Studio
\begin{verbatim}
Open Tools | Options | Projects and Solutions | VC++ Project Settings
Add ".cu" to the Extensions To Include (VS2010)
\end{verbatim}
Next, add CUDA include path to the project's VC++ Directories
\begin{verbatim}
Open the project's properties dialog
Select the Configuration Properties | VC++ Directories tab
Append $(CUDA_INC_PATH) to the list of Include Directories
\end{verbatim}
example: \verb!"$(IncludePath);$(CUDA_INC_PATH)".! Adding the CUDA include path
will allow VS to find included CUDA headers, line "cuda.h" in the IDE. Without
this step CUDA projects will still compile but right clicking on a CUDA header
file and selecting Open Document will not work. If you are using other libraries
like Thrust then you will want to make sure that it's include path is also added
here.   
\url{http://www.ademiller.com/blogs/tech/2010/10/visual-studio-2010-adding-intellisense-support-for-cuda-c/}

Just decide whether you want to develop your kernels on top of the driver only,
or the runtime. Then, you just use nvcc to compile your kernels (written in
C/C++) to a .cubin, and P/Invoke the driver (or runtime) DLL functions that
allocate memory, launch the kernels, etc.
\begin{verbatim}
[DllImport("nvcuda")]

public static extern CUResult cuMemAlloc(ref CUdeviceptr dptr, uint bytesize);
\end{verbatim}  

\section{managedCUDA (from CUDA 6.5)}
\label{sec:managedCUDA}

ManagedCUDA provides a complete wrapper for CUDA Driver API (version 6.5) in
C\#. CUDA C APIs provides the CUDA runtime APIs which wrap CUDA Driver APIs.
Thus, to wrap CUDA Driver APIs in an object-oriented, we need to use
ManagedCUDA. In general you can find C\# classes for each Cuda handle in the
driver API.
\begin{verbatim}
CUContext   -->  CudaContext class
\end{verbatim}
 In the original Cuda driver API those are given by standard C pointers. In
managedCuda these are represented by the class
\verb!Cuda[Pitched]DeviceVariable<T>! which is a generic class allowing type
safe and object-oriented access to CUDA Driver APIs.


We can use CUDA context, kernel, device variable, etc. directly from C\#.
There are different binaries to work with different .NET versions: .NET 2.0,
.NET 4.0 and Mono/Linux. The default one is for .NET 4.0.

It also provides wrapper for graphics interop with DirectX, OpenGL via SlimDX
and OpenTK, respectively. 

Operators (+, -, *, /) and proper \verb!ToString()! methods for CUDA vector
types: int2, float3, \ldots

Wrappers to all CUDA libraries: CUFFT, CURAND, CUSPARSE, CUBLAS, NPP.

\subsection{Allocate and Initialize data in GPU}

One line to do allocation and initialization: \verb!CudaDeviceVariable! generics
is used to define a variable of a specific type on GPU.
\begin{verbatim}
float3[] array_host = new float3[100];
for (int i = 0; i < 100; i++)
{
  array_host[i] = new float3(i, i+1, i+2);
}

// copy from CPU to GPU
//alloc device memory and copy data:
CudaDeviceVariable<float3> array_device = array_host;

// copy from GPU to CPU
//alloc host array and copy data: 
float3[] array_host2 = array_device; 
\end{verbatim}

 \verb!CudaDeviceVariable! generics accept any user defined tpe if it is a value
 type, i.e. a struct in C\#.

\subsection{Access GPU array elements using [] operator}

\begin{verbatim}
CudaDeviceVariable<float> devVar = new CudaDeviceVariable<float>(64);
devVar[0] = 1.0f;
devVar[1] = 2.0f;
float hostVar1 = devVar[0];
float hostVar2 = devVar[1];
\end{verbatim}

\subsection{NPPs extension methods can be used}

Add a reference to NPP library (Chap.\ref{chap:nvidia-npp}) and include
\verb!ManagedCuda.NPP.NPPsExtensions! namespace. 


\section{CUDAfy.NET}
\label{sec:CUDAfy}

CUDAfy allows .NET developers to easily create complex applications that split
processing cleanly between host and GPU. CUDAfy supports both CUDA and OpenCL
code generation and therefore has the ability to run the same applications on:
\begin{enumerate}
  \item CUDA-capable GPU (using CUDA or OpenCL)
  \item AMD GPU (using OpenCL)
  \item Intel CPU (using OpenCL)
\end{enumerate}

CUDAfy.NET (1.27) currently supports CUDA 6.0, with CUDA math libraries (CUFFT,
CUBLAS, CUSPARSE, CURAND). Using these math libraries is not supported with CUDA
5.x and 6.5.

\url{https://cudafy.codeplex.com/downloads/get/882502}

CUDAfy.NET has examples and supporting documents; yet the binaries requires .NET
4.0 or above. Also, CUDAfy.NET still lacks some CUDA features (texture
management, etc.)

Code:
\url{http://www.codeproject.com/Articles/202792/Using-Cudafy-for-GPGPU-Programming-in-NET}

\section{CUDA.NET}
\label{sec:CUDA.NET}

CUDA.NET is old (support upto CUDA 3.0), with little support document, and no
example.
\begin{enumerate}
  \item CUDA.NET 1.1: support CUDA 1.1 only. 
  
  (alpha) Windows XP only, (beta) add Linux support, (beta 2) add Windows Vista,
  (beta 3) CUDADriver, CUBLASDriver and CUDA structures, MONO 1.9, add Mac OS/X
  support, (official) add object model to ease development and 32/64-bit O/S.
  
  \item CUDA.NET 2.0: support CUDA 2.0: double precision, new BLAS routines
  
  \item CUDA.NET 2.0.1: add \verb!ToString()! method in CUDAException
  
  \item CUDA.NET 2.0.2: use C\# reflection in CUDA, CUFFT, and CUBLAS classes
  (Sect.\ref{sec:reflection}).
  
  \item CUDA.NET 2.0.3: add CUDA runtime API, Direct3D, OpenGL interoperability,
  add texture function, add 3D memory operations, fix copying vector types.
  
  \item CUDA.NET 2.0.4: add support CUFFT and CUBLAS emulation, add new class
  CUDAExecution for easy execution CUDA kernels
  
  add Windows HPC Server 2008, Windows CE.
  
  \item CUDA.NET 2.1 (2009): support CUDA 2.1 APIs (DirectX 10 and JIT compiler
  support)
  
  \item CUDA.NET 2.2: support CUDA 2.2 (zero-copy, extended APIs)
  
  \item CUDA.NET 2.3: support CUDA 2.3 (double precision FFT, extended OpenGL
  APIs, fixed runtime APIs)
  
  CUDA.NET 2.3.1: fixed FFT naming to match CUFFT
  
  CUDA.NET 2.3.2: update DirectX and OpenGL APIs to conform CUDA APIs
  
  CUDA.NET 2.3.3: minor update to CUDA class
  
  CUDA.NET 2.3.4/2.3.5: update native wrapper, remove special marshalling
  directives (Sect.\ref{sec:marshalling})
  
  CUDA.NET 2.3.6: update native wrappers, add \verb!SizeT! structure to handle
  32/64 systems compatibility for functions taking \verb!size_t! as parameter.
  
  CUDA.NET 2.3.7: update runtime types to work with \verb!SizeT!
  
  \item CUDA.NET 3.0.0: support CUDA 3.0 APIs, add special class for CUDA
  synchronization in multi-device and multi-threaded applications.
  
  add Windows 7 support.
\end{enumerate}
 

\url{http://www.cass-hpc.com/solutions/libraries/cuda-net/cuda-net-archive}


\section{GPU.NET}
\label{sec:GPU.NET}

