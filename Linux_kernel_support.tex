\chapter{Linux kernel support for CPU and GPU}

\section{Linux kernel 4.14: Heterogeneous Memory Management}
\label{sec:HMM-shared-memory-space-GPU-CPU}

Heterogeneous Memory Management (HMM) has also finally made it into Linux. This
feature enables GPUs and CPUs to access a process's shared address space. Both
supercomputing and programs using parallel-processing languages and application
programming interfaces (API) such as OpenCL 2.0, Nvidia's CUDA, and C++'s
Parallel Algorithm Scheduling Library (PASL) will find this useful.

\url{https://git.kernel.org/pub/scm/linux/kernel/git/torvalds/linux.git/commit/?id=bffc33ec539699f045a9254144de3d4eace05f07}

Traditionally, a (GPU) driver expose dedicated memory allocation API through
their device file, often relying on a combination of IOCTL and mmap calls.
On the other hand, memory allocated using (malloc, mmap of a file, share memory,
etc.) only accessible by CPU. 

The only memory regions that are accessible by CPU and (GPU) drivers is known as
{\bf pinned} memory. 

