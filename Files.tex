
\chapter{Files}
\label{chap:files}



% \section{Open a file}
% \label{sec:open-file}



% \section{Write to a file}
% \label{sec:write-file-1}



% \section{Read from a file}
% \label{sec:read-from-file}

\section{End-of-File detection}
\label{sec:end-file-detection}

We can use the END= keyword in the READ statement with the argument of
END= is the statement label which will be executed when the
End-of-File occur.

End-of-File detection is useful when reading a file of unknown length,
but some cautions is necessary. Especially when reading several record
at a time from a formatted file, the data list items beyond the
End-of-File will have their values unaltered. 


\section{Check file status}
\label{sec:check-file-status}

Commonly, before performing some operations on a file, we want to know
in advance its status. In Fortran, this can be done with the INQUIRE
statement. The file is specified via the FILE clause (if the file has
not opened, which require a string or a CHARACTER object containing
the file name) or the UNIT clause (which is a file identifier of an
opened file)\footnote{\url{http://rosettacode.org/wiki/File_Exists}}.

\begin{verbatim}
INQUIRE ([UNIT = fileref, ] inquiry-list)
INQUIRE ([FILE = fileref,] inquiry-list)
\end{verbatim}

The information are returned in the dummy arguments specified in the
$inquiry-list$ which is the set of different clauses
\begin{lstlisting}
IOSTAT = int_var    ! 0 for success; otherwise 
   !it get a processor-dependent possitive number 
EXIST = log_var     ! TRUE if file exist
OPENED = log_var    ! TRUE if file opened
NUMBER = int_var    ! the file identifier if file is 
         opened, otherwise an UNDEFINED value
NAMED = log_var     ! TRUE if the file has a name
NAME = char_var     ! the name of the file if file is 
         named, otherwise an UNDEFINED value
\end{lstlisting}

{\bf IMPORTANT}: INQUIRE is the statement, not a subroutine, so we don't
need to use CALL in front of it. Further, remember to explicitly use
FILE and argument name, e.g. EXIST, OPENED... when calling the
statement. 
\begin{lstlisting}
INQUIRE(filename, EXIST=ext)      ! wront

INQUIRE(FILE=filename, EXIST=ext) ! correct
\end{lstlisting}


\section{Remove a file}
\label{sec:remove-file}

From inside the program, we can call system commands, by representing the
command as a string and pass it to the SYSTEM subroutine.

\begin{lstlisting}
str = 'rm readme.txt ' ! remove the file readme.txt
CALL SYSTEM (str)
\end{lstlisting}

If the file is to be used for another purpose right immediately, you
can choose the option to open it and replace it.
\begin{lstlisting}
OPEN(nout,file='x.txt',STATUS='replace')
\end{lstlisting}

\section{Reposition the file cursor}
\label{sec:repos-file-curs}

\subsection{REWIND}
\label{sec:rewind}


Use the REWIND statement to reposition at the beginning of the file
for the next READ or WRITE. Generally you use a simple form like
"rewind (11)", but you can include IOSTAT= and ERR= options. 

\subsection{BACKSPACE}
\label{sec:backspace}

 At times during fancy input processing, you need to read a record,
 process some information, then re-read the record to absorb the full
32 information. This lets you back up one record so it can be
 re-read. Repeated use of BACKSPACE is permitted to backup more than
 one record. 

\section{Obsolete features}
\label{sec:obsolete-features-2}


\begin{itemize}
\item ENDFILE
\end{itemize}

%%% Local Variables: 
%%% mode: latex
%%% TeX-master: "gpucomputing"
%%% End: 
