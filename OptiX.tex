\chapter{OptiX Ray Tracing Engine}
\label{chap:OptiX}


\section{Introduction to Ray Traching}

As rendering technology develops, its becoming difficult to distinguish from
real worlds and virtual worlds. The situation is
\begin{itemize}
  \item You have a scene filled with various objects and light sources illuminating them.

  \item You have a virtual camera located at a particular position.

  \item Rendering basically captures the scene as the camera would and
  generate an image.
\end{itemize}

Rendering is the process of generating an image (i.e. a {\it rendering}) from a
2D or 3D model (a model is collectively called a {\it scence} file), using a
computer program. We need to use some language or data structure to define this
scence file which contains objects with geometry, viewpoint, texture,
lighting, and shading information as a description of the virtual scene. 

Once the light source is considered, we need to incorporate the effect of
light/shading on the final image. The light-transport modelling
techniques can be classified in 3 groups
\begin{enumerate}
  \item rasterization, e.g. scanline rendering: just projecting the objects in
  the scence to a 2D image plane, without advanced optical effects
  
  \item ray casting: some basic optical laws of reflection intensity can be used
  
  \item ray tracing: more advanced than ray casting, with advanced optical
  effects, usually with Monte carlo techniques to obtain more realistic results.
  However, the speed is often orders of magnitude slower.
  
  \item radiosity: use to calculates the passage of light as it leaves the light
  sources and illuminate surfaces.
\end{enumerate}

{\bf Ray traching} is the very simple and elegant rendering
method. Ray tracing treats light as consisting of straight rays emanating from a light
source and being reflected from.


\subsection{OptiX 3.0.1}

\begin{verbatim}
C:\Program Files\NVIDIA Corporation\OptiX SDK 3.0.1
\end{verbatim}

\section{Radiosity}
\label{sec:radiosity}  

Unlike rendering methods that use Monte Carlo algorithms (such as path tracing),
which handle all types of light paths, typical radiosity methods only account
for paths which leave a light source and are reflected diffusely some number of
times (possibly zero) before hitting the eye; such paths are represented by the
code "LD*E"   

Radiosity is a global illumination algorithm in the sense that the illumination
arriving on a surface comes not just directly from the light sources, but also
from other surfaces reflecting light  

\url{http://en.wikipedia.org/wiki/Radiosity_(computer_graphics)}

\url{http://mynameismjp.wordpress.com/2011/01/31/radiosity-dx11-style/}

\url{https://optixdotnet.codeplex.com/}