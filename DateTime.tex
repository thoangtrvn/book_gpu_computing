%%
%% DateTime.tex
%% Login : <hoang-trong@hoang-trong-laptop>
%% Started on  Thu Jul 16 14:06:20 2009 Hoang-Trong Minh Tuan
%% $Id$
%% 
%% Copyright (C) 2009 Hoang-Trong Minh Tuan
%%

\chapter{Date Time}
\label{chap:date-time}


In this chapter, we will discuss important concepts about timing when
designing a benchmark and also different subroutine/function to tell
the date/time. 


\section{Types of times}
\label{sec:types-times}


x86 CPUs provide 4 protection rings
(0,1,2,3)\footnote{\url{http://en.wikibooks.org/wiki/Windows_Programming/User_Mode_vs_Kernel_Mode}}.
The two most common modes are {\it user mode} and {\it kernel mode}:
\begin{itemize}
\item In kernel mode (ring 0 - maximum performance), the code can get
  access to every resource. A code need to be in kernel mode to use
  system calls.
\item In user mode (ring 3 - maximum stability), the code has
  restricted access to resources.
\end{itemize}
Transition between user mode and kernel mode is expensive. That's why
programs that throw exception are slow, for
example\footnote{\url{http://www.codinghorror.com/blog/archives/001029.html}}.

Correspondingly, there are three types of times:
\begin{itemize}
\item {\bf User time}: total amount of time the CPU spending to run
  user code (non-system calls), i.e. excluding the time the CPU
  spending on the kernels (i.e. system calls like {\it exec} or
  {\it folk}) or delay on I/O.
\item {\bf System time}: the time that the code spends executing
  system calls
\item {\bf CPU time}: total amount of time the CPU spending to run your code
  or anything else requested by your code, including the time the CPU
  spending on the kernels. If you have more than one CPU, the time for
  each is added. 
\item {\bf Wall-clock time}: time taken to run your code from start to
  end. This is equivalent to the time when you're timing your code
  with a stopwatch. Thus this time can be affected by anything else,
  like waiting for I/O, CPU is used by other programs. It includes
  three terms: CPU time, I/O time, communication channel delay.
\end{itemize}
In general, for a sequential program, we will have
{\it usertime $<$ CPU time $<$ wall-clock time} and
\begin{eqnarray*}
  \text{wall-clock time} \ge \text{user time} + \text{system time}
\end{eqnarray*}
However, this can be different for a parallel program running on
multiple cores.

With the \verb.time. command, we can have
\begin{verbatim}
!!! sequential code
real  0m0.009s    ! wall-clock
user  0m0.001s    ! user time
sys  0m0.002s    ! system time
!!! parallel (4 cores)
real  40m12.994s 
user  157m8.318s  ! almost 4 times of wall-clock
                    ! as it is the total time of 4CPUs
sys  0m1.486s
\end{verbatim}
% Time measurements using in a benchmark are usually defined as the
% starting time and the ending time, the difference between them is
% called the elapsed ``wall-clock'' time. However, during running the
% program, it can not utilize the CPU all the time, but there are other
% services or programs using the CPU also. Thus, the total time that the
% program use the CPU is called the CPU time.

References:
\begin{itemize}
\item \url{http://people.sc.fsu.edu/~burkardt/f_src/timer/timer.html}
\item
  \url{http://gwyn.tux.org/~balsa/linux/benchmarking/articles/html/Article1d-1.html}
\item \url{http://www.ibiblio.org/pub/languages/fortran/ch3-10.html}
\end{itemize}

\subsection{Wall-clock time}
\label{sec:wall-clock-time}


Wall-clock time is the real-time system clock. It is different from
CPU time, as it doesn't care whether the program/subprogram use the
CPU or not. The ``wall-clock'' time is important in parallel
programming, as we want to know both the CPU time and the time waiting
for I/O or any other service. This can be obtained using

\subsubsection{OpenMP}
\label{sec:openmp-2}


\begin{lstlisting}
seconds = omp_get_wtime ( )
   ///operations to time;
seconds = omp_get_wtime ( ) - seconds;
\end{lstlisting}


\subsubsection{MPI}
\label{sec:mpi-2}


\begin{lstlisting}
seconds = MPI_Wtime ( );
   ///do something here
seconds = MPI_Wtime ( ) - seconds;
\end{lstlisting}

\subsubsection{MatLab}
\label{sec:matlab}

Use tic and toc
\begin{verbatim}
tic;
  ///do something here
seconds = toc;
\end{verbatim}



\subsection{CPU Time}
\label{sec:cpu-time}

CPU time is the amount of elapsed computer time that was used by your
program. So, this will count only the time that your program use the
CPU. This is different from ``real'' time (or ``wall-clock'' time)
which calculate the time since the program/subprogram start until it
finish, including the time it waits for CPU, I/O, or other system
functions. 

With parallel programming, it is, however, more important to know the
elapsed ``wall-clock'' time. You can go back to the previous section
for this.



The previous sections deal with real-time clock which is different
from CPU time.  To know the CPU time (in seconds) taken by the current
process (counting also all of its child processes) from the start of
the program, you use the standard intrinsic \verb!CPU_TIME!. The
difference with those using real-time clock is that it returns the
time that the program is actually running, and not the time that the
program is suspended or waiting. This is a more precise method to
benchmark a program since there can be other programs are running on
the CPU also.

% Since Fortran 95, you can use this useful subroutine for testing
% segments of code to determine the execution time (benchmarking).
\begin{lstlisting}
CALL CPU_TIME(time)
\end{lstlisting}
with the argument {\it time} (measured in seconds) is REAL,
INTENT(OUT). The returned value varies depending upon the setting of
XLFRTEOPTS environment variable. This can be done via the command line
or using the subroutine {\bf setrteopts}, e.g.
\begin{lstlisting}
call setrteopts('cpu_time_type=alltime')
call func()
call CPU_TIME(T1)
\end{lstlisting}
This environment variable is checked only once and at the starting of
the program. This, however, applies to XL-Fortran only.

By taking the difference for the CPU time at two different points, we
can know the running time. An alternate way is to use etime.

Obsolete functions (not recommend to use): \verb!SECOND! (both
subroutine and function)


\section{Date \& Time}
\label{sec:date--time}


To get the current date and time in Fortran, you need 2 arrays of type
INTEGER*4, each with 3 elements.
\begin{itemize}
\item idate returns an array with day, month, and year.
\item itime returns an array with hour, minute, and second.
\end{itemize}

\begin{lstlisting}
INTEGER*4 today(3), now(3)
  ! today(1)=day, (2)=month, (3)=year
CALL idate(today)   
  ! now(1)=hour, (2)=minute, (3)=second
CALL itime(now)     

WRITE ( *, "('Date ', i2.2, '/', i2.2, '/', i4.4, &
  '; time ', i2.2, ':', i2.2, ':', i2.2)" )  &
   today(2), today(1), today(3), now
\end{lstlisting}
And output is 
\begin{verbatim}
Date 04/01/1995; time 14:07:40
\end{verbatim}

Since Fortran 95, you have another subroutine
\begin{lstlisting}
DATE_AND_TIME(DATE, TIME, ZONE, VALUES)
\end{lstlisting}
which returns the date and time from the wall-clock time,
i.e. real-time clock. All are INTENT(OUT) arguments. The first three
arguments are CHARACTERS, at least 8, 10 and 5 characters long,
respectively; the fourth argument is INTEGER, DIMENSION(8).
\begin{itemize}
\item  DATE is in form $ccyymmdd$, 

\item TIME is in form $hhmmss.sss$, 

\item ZONE is in form $(\pm)hhmm$

\item VALUES is an array of 8 elements

  \begin{itemize}
  \item   VALUE(1):   The year
\item  VALUE(2):   The month
\item  VALUE(3):   The day of the month
\item   VALUE(4):   Time difference with UTC in minutes
\item   VALUE(5):   The hour of the day
\item   VALUE(6):   The minutes of the hour
\item   VALUE(7):   The seconds of the minute
\item VALUE(8): The milliseconds of the second
  \end{itemize}
\end{itemize}

\section{SYSTEM clock}
\label{sec:system-clock}

Different CPU and compiler have different fast clock. This clock is
ticking from the start of the program, and is at a given rate
(i.e. number of ticks per second). The value of this counting rate is
given by the output \verb!CLOCK_RATE!.


Since Fortran 95, to determine the number of ticks per second in the
wall-clock time, you use the subroutine {\bf SYSTEM\_CLOCK}. It counts
times at a rate of \verb!count_rate! (counts per second); when it
reaches \verb!count_max!, the count resets to zero and resume
counting. 
\begin{lstlisting}
SYSTEM_CLOCK ([count, count_rate, count_max])
\end{lstlisting}
with all arguments are optionals and of type  INTEGER, INTENT(OUT).

This subroutine returns the {\it count} in milliseconds of the system
clock time since the Epoch (00:00:00 UTC, Jan 1, 1970). {\it
  count\_rate} and {\it count\_max} are constant and specific to
gfortran.

If there is no clock, the count is set to $-HUGE(count)$, and the
other two quantities are zero.



\section{etime}
\label{sec:etime}

Up until Fortran 90, there is no intrinsic procedure to measure CPU
time. We, however, can use {\bf etime} which is a UNIX system
command\footnote{In XL-Fortran, it is defined in the {\it xlfutility}
  module}.
Thus, {\it etime} is not part of the language and cannot be use in
other systems.

\begin{lstlisting}
use xlfutility
\end{lstlisting}
You can check how long your program have been running, regardless what
else is running on the same processor. The time reports in seconds, in
3 parts:
\begin{itemize}
\item  User time, time actually spent in your program;
\item  System time, time spent in the operating system on your
  program's behalf; and
\item  Total time. 
\end{itemize}

{\bf etime} needs 2 output arguments: the first one returns the {\it
  user-time}, the second one returns the {\it system-time}, and the
function itself return the {\it total-time}.
\begin{lstlisting}
      real etime          ! Declare the type of etime()
      real elapsed(2)     ! For receiving user and system time
      real total          ! For receiving total time

      total = etime(elapsed)
      print *, 'End: total=', total, ' user=', elapsed(1),
     &         ' system=', elapsed(2)
\end{lstlisting}

\section{Time in seconds}
\label{sec:time-seconds}

You can get the number of seconds elapsed since a given time point $X$
(in seconds also) from the real-time system clock using a non-standard
function (GNU extension) \verb!SECNDS(X)!  with $X$ is a reference
time (in seconds). If $X=0$, the time, in seconds, calculated from
mid-night is returned. It measure the number of seconds of maximum
24hrs. 

Thus, to calculate the runtime for a piece of code, you can use
\begin{lstlisting}
program test_secnds
integer :: i
real(4) :: t1, t2
  print *, secnds (0.0)   ! seconds since midnight
  t1 = secnds (0.0)       ! reference time
  do i = 1, 10000000      ! do something
  end do
  t2 = secnds (t1)        ! elapsed time
  print *, "Something took ", t2, " seconds."
end program test_secnds
\end{lstlisting}

%%% Local Variables: 
%%% mode: latex
%%% TeX-master: "gpucomputing"
%%% End: 
