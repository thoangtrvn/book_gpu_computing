
\chapter{Patch a program}
\label{chap:patch-program}

To compare the difference between two files, you can use different
tools.
\begin{itemize}
\item To apply a patch to a source code, you need to have a file that
display the difference between the current version and the new
version.  This can be done by the \verb!diff! tool.

\item To compare two compressed files, you can use either \verb!zcmp!
  or \verb!zdiff! or \verb!bzdiff!. 

\item To merge two files,  you can use \verb!sdiff!

\item To display word difference between two files, you can use
  \verb!wdiff!. 
\end{itemize}

\section{diff tool}
\label{sec:diff-tool}

diff tool is a {\bf file comparison utility} that output the
difference between two files. The files can be text files or binary
files. For text files, it display changes made per line. The output is
called a {\bf patch}. To fix the current source file, you can apply
the patch to the current version using a Unix tool called
\verb!patch!.

Here is a sample patch file:
\begin{verbatim}
--- /etc/default/grub 2009-11-12 17:44:41.484057760 -0500
+++ /tmp/grub.4eVHSnVuOQ 2010-02-10 15:22:03.327081665 -0500
@@ -4,7 +4,7 @@
 GRUB_DEFAULT=0
 #GRUB_HIDDEN_TIMEOUT=0
 GRUB_HIDDEN_TIMEOUT_QUIET=true
-GRUB_TIMEOUT=10
+GRUB_TIMEOUT="10"
 GRUB_DISTRIBUTOR=`lsb_release -i -s 2> /dev/null || echo Debian`
 GRUB_CMDLINE_LINUX_DEFAULT="quiet splash"
 GRUB_CMDLINE_LINUX=""
\end{verbatim}
At the beginning of the patch file is the two-line headers: the
original files is preceded by ``---'', while the new file is preceded
by ``+++''. Each line contain the file information, including the full
path and time stamp. 
\begin{verbatim}
--- /path/to/original ''timestamp''
+++ /path/to/new      ''timestamp''
\end{verbatim}

After the two header lines, there are one or more {\bf change hunks}
(chunks) that contain the line difference between the two files. A
chunk begins with a range of information (surrounded by a double-at
signs @@)
\begin{verbatim}
@@ -R +R @@
\end{verbatim}
with $R$ is of the form $l,s$, where $l$ is the starting line number,
and $s$ is the number of lines the change hunk applies to. Here $-R$
is the chunk range for the old file, and $+R$ is the chunk range for
the new file. 


, then followed by lines starting with either
\begin{itemize}
\item The ``!'' represent a change between line in two files

\item The ``+'' represents the addition of a line

\item The ``-'' represents the deletion of a line
\item The blank space represent an unchanged line 
\end{itemize}
%%% Local Variables: 
%%% mode: latex
%%% TeX-master: "gpucomputing"
%%% End: 
